%
% Tesi D.S.I. - modello preso da
% Stanford University PhD thesis style -- modifications to the report style
%
%%%%%%%%%%%%%%%%%%%%%%%%%%%%%%%%%%%%%%%%%%%%%%%%%%%%%%%%%%%%%%%%%%%%%%%%%%%
%                                                                         %
%			TESI DOTTORATO                                                   %
%			______________                                                   %
%                                                                         %
%			AUTORE: Elena Pagani                                             %
%                                                                         %
%			Ultima revisione: 7.X.1998                                       %
%           correzioni atrent                                             %
%%%%%%%%%%%%%%%%%%%%%%%%%%%%%%%%%%%%%%%%%%%%%%%%%%%%%%%%%%%%%%%%%%%%%%%%%%%
%
%[twoside,openright]
\documentclass[a4paper,12pt,twoside,openright]{report}
%    \renewcommand{\baselinestretch}{1.6}      % interline spacing
%
% \includeonly{}
%
%			PREAMBOLO
%
\usepackage[a4paper]{geometry}
\usepackage[italian]{babel}
\usepackage{amssymb,amsmath,amsthm}
\usepackage{graphicx}
\usepackage{url}
\usepackage{epsfig}
\usepackage{setspace}
\usepackage{tesi}
\usepackage{amssymb}
\usepackage{amsmath}
\usepackage[pdfa]{hyperref}
\usepackage{color,soul}
\usepackage{float}
\usepackage{listings}
\usepackage{tikz}
\usepackage{graphicx}
\usepackage{caption}
\usepackage{mathtools}
\usepackage{nameref}
\usepackage[a-1b]{pdfx}


\usepackage[normalem]{ulem} % [normalem] prevents the package from changing the default behavior of `\emph` to underline.

\usetikzlibrary{positioning}

% per le accentate
\usepackage[utf8]{inputenc}
%
\newtheorem{myteor}{Teorema}[section]
%
\newenvironment{teor}{\begin{myteor}\sl}{\end{myteor}}
%
%
%			TITOLO
%
\begin{document}

\newcommand{\E}{È}
%\noindent
%\hl{\textbf{Scadenze:}\\
%\textbf{Presentazione domanda e titolo:} entro il 22 Novembre\\
%\textbf{Consegna riassunto:} dal 22 novembre al 6 dicembre\\
%\textbf{Consegna tesi:} entro il 6 dicembre\\
%\textbf{Seduta laurea:} dal 13 al 17 dicembre (ancora da definire)
%}
\title{Uso di un modello BERT per la correzione di errori generati dal processo di OCR su dati testuali}
\author{Sebastiano Caccaro}
\dept{Corso di Laurea Magistrale in Informatica} 
\anno{2020-2021}
\matricola{958683}
\relatore{Prof. Alfio FERRARA}
\correlatore{Dr. Francesco PERITI}
%
%        \submitdate{month year in which submitted to GPO}
%		- date LaTeX'd if omitted
%	\copyrightyear{year degree conferred (next year if submitted in Dec.)}
%		- year LaTeX'd (or next year, in December) if omitted
%	\copyrighttrue or \copyrightfalse
%		- produce or don't produce a copyright page (false by default)
%	\figurespagetrue or \figurespagefalse
%		- produce or don't produce a List of Figures page
%		  (false by default)
%	\tablespagetrue or \tablespagefalse
%		- produce or don't produce a List of Tables page
%		  (false by default)
% 
%			DEDICA
%
\beforepreface
\prefacesection{}
\begin{flushright}
\Large {\sl "Everything's in order in a black hole"}\\
\sl \large Arctic Monkeys
\end{flushright}
% 
%			PREFAZIONE
%
\prefacesection{Sommario}
Ad oggi, sempre più libri cartacei, riviste e giornali presenti in biblioteche e archivi storici stanno venendo trasformati in versioni elettroniche che possono essere manipolate da un computer. A questo scopo, nel corso degli anni sono state sviluppate tecnologie di Optical Character Recognition (comunemente abbreviato con OCR) per tradurre le scansioni e immagini di documenti testuali in testo interpretabile e processabile da un computer. Questi sistemi, però, non sono perfetti e possono introdurre errori nel testo, che possono abbassare drasticamente la precisione di vari task di NLP. In questa tesi si propone un approccio modulare per la correzione di tali errori basato su BERT, un modello di machine learing finalizzato al NLP rilasciato da Google nel 2018. L'approccio proposto è valutato su un dataset creato ad hoc, contenente testi con diverse tipologie e intensità di errore.  I risultati sperimentali dimostrano la fattibilità di utilizzare approcci basati su BERT per la correzione di errori in testi acquisiti tramite OCR, mostrando inoltre come ci siano ulteriori margini per migliorare l'approccio proposto.
%
%
%			RINGRAZIAMENTI
%
\prefacesection{Ringraziamenti}
{\sl


Innanzitutto, vorrei ringraziare il Prof. Ferrara, relatore di questa tesi, e il mio correlatore, il Dott. Periti, che mi hanno guidato egregiamente nello svolgimento di questo elaborato, nonostante le difficoltà derivate dal non potersi incontrare di persona.\\

Un ringraziamento speciale alla mia famiglia, in particolare a mia madre e mio padre, che non mi hanno fatto mai mancare il supporto e i mezzi per portare a termine gli studi.
\\

Un super ringraziamento va al Pav, sempre presente nel bene e nel male, con la pioggia e con la neve, con i tuoni e con fulmini, ma non esageriamo sennò si monta la testa.\\
Un ringraziamento speciale va anche ai SUONI aka Dido, Bajo e Vitto perchè alla fine, se porgi il bicchiere per fare cin-cin, sicuramente all'altra persona parte il riflesso di farlo anche lui.\\
Impossibile non ringraziare anche la Marghe per le margheritate durante e dopo la Via degli Dei, e la Greta, che mi aggiorna sempre sulle condizioni di Davide.\\

Ringrazio inoltre il Jack, i $\delta$-cappuccio, Roberto Pirelli e i compagni della puglia, e tutti gli amici che, anche se indirettamente, hanno contribuito a non farmi andare in esaurimento prima, durante e dopo il lockdown.\\

Ringrazio infine Filippo, anche se probabilmente ha di meglio da fare.
}
\afterpreface
% 
% 
\chapter*{Introduzione}
\addcontentsline{toc}{chapter}{Introduzione}  
\label{sec:intro}
\label{sec:arte_intro}
Ad oggi, sempre più libri cartacei, riviste e giornali presenti in biblioteche e archivi storici stanno venendo trasformati in versioni elettroniche che possono essere manipolate da un computer. A questo scopo, nel corso degli anni sono state sviluppate tecnologie di Optical Character Recognition (comunemente abbreviato con OCR) per tradurre le scansioni e immagini di documenti testuali in testo interpretabile e processabile da un computer. Questi sistemi, però, non sono perfetti e possono introdurre errori nel testo. Può accadere, infatti, che durante il processo di scansione alcuni caratteri vengano letti in modo errato, altri vengano aggiunti e altri ancora non riconosciuti. \E, ad esempio, particolarmente probabile che caratteri o sequenze di caratteri graficamente simili come \textit{"li"} e \textit{"n"} vengano scambiati fra di loro\cite{ocr_error_analysis}. La frequenza di tali errori è influenzata da fattori quali la condizione di deterioramento di un documento e la qualità di acquisizione dell'immagine\cite{hartley1999quality}: la presenza di granelli di polvere, caratteri scoloriti, pagine ingiallite o artefatti risultati dalla scansione, ad esempio, influiscono negativamente sulle performance dei sistemi OCR.\\
La presenza di tali errori in corpora acquisiti tramite OCR risulta problematica in quanto rende meno precisi task di Natural Language Processing (NLP) come, ad esempio, l'esecuzione di query \cite{impatto_ocr_1} o il topic modelling\cite{impatto_ocr_2}. Per ovviare a tali problemi, sono state sviluppate varie soluzioni che mirano a minimizzare il quantitativo di errori presenti nel testo estratto. \E\ possibile classificare queste soluzioni nelle seguenti due categorie:
\begin{itemize}
\item \textbf{OCR Pre-processing}: ricadono in questa categoria tutte quelle tecniche che mirano ad ottenere migliori risultati dall'estrazione del testo attraverso il miglioramento dell'input, ovvero delle immagini, che viene usato dai software di OCR. Tali metodi includono, ma non si limitano a, l'uso di migliori tecniche di scansione, la correzione del contrasto nell'immagine\cite{holley2009good} e la rotazione e correzione di deformazioni nell'immagine\cite{bieniecki2007image} (\autoref{fig:art_prep_ex}).

 
\item \textbf{OCR Post-processing}: ricadono in questa categoria tutte quelle tecniche che mirano ad individuare e correggere gli errori presenti nell'output generato dai vari software di OCR. Essendo l'OCR Post-processing oggetto di questa tesi, sarà approfondito a parte nella \autoref{sec:art_post_post}.
\end{itemize}

OCR pre-processing e post-processing sono spesso usati in congiunzione per ottenere migliori risultati dall'estrazione del testo.
\begin{figure}[H]
\centering
{
\begin{minipage}{0.35\textwidth}
\includegraphics[width=\textwidth]{immagini/stato_arte/prep1}
\end{minipage} 
\begin{minipage}{0.06\textwidth}
\centering
\Large$\rightarrow$
\end{minipage}
\begin{minipage}{0.35\textwidth}
\includegraphics[width=\textwidth]{immagini/stato_arte/prep2}
\end{minipage}
\caption{A sinistra foto di una pagina contenente del testo. A destra, foto della stessa pagina pre-processata per facilitare l'estrazione del testo. Esempio preso da \cite{bieniecki2007image}.}
\label{fig:art_prep_ex}
}
\end{figure}
\noindent

\paragraph{Scopo e organizzazione della tesi} Lo svolgimento di questa tesi è consistito nello sviluppo di un sistema di OCR post-processing basato su BERT, un modello di machine learning finalizzato al Natural Language Processing sviluppata da Google. Lo sviluppo di tale sistema ha necessitato della creazione di un dataset apposito, nonché di un'adeguata metodologia di test. Il codice prodotto, che permette di replicare i risultati illustrati in questa tesi, è disponibile al seguente link:
\begin{center}
\url{https://github.com/sebacaccaro/codice_tesi}
\end{center}
\noindent
La tesi è organizzata nei seguenti capitoli:

\begin{enumerate}
\item \textbf{\nameref{sec:arte}}: in questo capitolo è definito in modo più preciso il problema dell'OCR post-processing, ed è presentata la discussione della letteratura.

\item \textbf{\nameref{sec:dataset}}: in questo capitolo si discutono le ragioni che hanno portato alla creazione di un dataset apposito, e si descrive la metodologia con la quale tale dataset è stato creato.

\item \textbf{\nameref{sec:metodologia}}: in questo capitolo si descrive la metodologia implementata dal sistema di correzione sviluppato.


\item \textbf{\nameref{sec:test}}: in questo capitolo è descritta la metodologia di test utilizzata, insieme alle metriche definite per la valutazione del sistema di correzione. Sono poi riportati e commentati i risultati dei test eseguiti


\item \textbf{\nameref{sec:analisi}}: in questo capitolo si analizzano le performance dei vari stadi del sistema di correzione, con lo scopo di capire quali fasi sono ottimizzate e quali invece hanno margini di miglioramento.
\end{enumerate}
\noindent
Sono inoltre riportate in chiusura le conclusioni tratte dai risultati e dall'analisi dell'errore, insieme ad alcuni possibili sviluppi futuri del sistema di correzione.










\chapter{Stato dell'Arte}
\label{sec:arte}
In questo capitolo si passa in rassegna lo stato dell'arte riguardante il tema dell'OCR post-processing.\\
Nella \autoref{sec:arte_intro} è presentata un' introduzione al problema affrontato e alcune definizioni di base. Nella \autoref{sec:art_post_post} sono trattati gli approcci più recenti al problema dell'OCR post-processing presenti in letteratura, divisi in base al tipo di metodologia adottata. Per ogni categoria si fornisce una breve introduzione atta a rendere più chiari gli approcci presentati.

\section{Introduzione}
\label{sec:arte_intro}
\paragraph{Optical Character Recognition}
Ad oggi, sempre più libri cartacei, riviste e giornali presenti in biblioteche e archivi storici stanno venendo trasformati in versioni elettroniche che possono essere manipolate da un computer. A questo scopo, nel corso degli anni sono state sviluppate tecnologie di Optical Character Recognition (comunemente abbreviato con OCR) per tradurre le scansioni e immagini di documenti testuali in testo interpretabile e processabile da un computer. Questi sistemi, però, non sono perfetti e possono introdurre errori nel testo. Può accadere, infatti, che durante il processo di scansione alcuni caratteri vengano letti in modo errato, altri vengano aggiunti e altri ancora non riconosciuti. \E, ad esempio, particolarmente probabile che caratteri o sequenze di caratteri graficamente simili come \textit{"li"} e \textit{"n"} vengano scambiati fra di loro\cite{ocr_error_analysis}. La frequenza di tali errori è influenzata da fattori quali la condizione di deterioramento di un documento e la qualità di acquisizione dell'immagine\cite{hartley1999quality}: la presenza di granelli di polvere, caratteri scoloriti, pagine ingiallite o artefatti risultati dalla scansione, ad esempio, influiscono negativamente sulle performance dei sistemi OCR.\\
La presenza di tali errori in corpora acquisiti tramite OCR risulta problematica in quanto rende meno precisi task di Natural Language Processing (NLP) come, ad esempio, l'esecuzione di query \cite{impatto_ocr_1} o il topic modelling\cite{impatto_ocr_2}. Per ovviare a tali problemi, sono state sviluppate varie soluzioni che mirano a minimizzare il quantitativo di errori presenti nel testo estratto. \E\ possibile classificare queste soluzioni nelle seguenti due categorie:
\begin{itemize}
\item \textbf{OCR Pre-processing}: ricadono in questa categoria tutte quelle tecniche che mirano ad ottenere migliori risultati dall'estrazione del testo attraverso il miglioramento dell'input, ovvero delle immagini, che viene usato dai software di OCR. Tali metodi includono, ma non si limitano a, l'uso di migliori tecniche di scansione, la correzione del contrasto nell'immagine\cite{holley2009good} e la rotazione e correzione di deformazioni nell'immagine\cite{bieniecki2007image} (\autoref{fig:art_prep_ex}).
\begin{figure}[H]
\centering
{
\begin{minipage}{0.35\textwidth}
\includegraphics[width=\textwidth]{immagini/stato_arte/prep1}
\end{minipage} 
\begin{minipage}{0.06\textwidth}
\centering
\Large$\rightarrow$
\end{minipage}
\begin{minipage}{0.35\textwidth}
\includegraphics[width=\textwidth]{immagini/stato_arte/prep2}
\end{minipage}
\caption{A sinistra foto di una pagina contenente del testo. A destra, foto della stessa pagina pre-processata per facilitare l'estrazione del testo. Esempio preso da \cite{bieniecki2007image}.}
\label{fig:art_prep_ex}
}
\end{figure}

 
\item \textbf{OCR Post-processing}: ricadono in questa categoria tutte quelle tecniche che mirano ad individuare e correggere gli errori presenti nell'output generato dai vari software di OCR. Essendo l'OCR Post-processing oggetto di questa tesi, sarà approfondito a parte nella \autoref{sec:art_post_post}.
\end{itemize}


OCR pre-processing e post-processing sono spesso usati in congiunzione per ottenere migliori risultati dall'estrazione del testo.

\section{OCR Post-processing}
\label{sec:art_post_post}
In letteratura sono presenti numerosi approcci al problema dell'OCR post-processing, molti dei quali adottano strategie molto differenti. Dato ciò, non è possibile delineare una metodologia generale che ogni approccio segue, ma, in generale, ogni approccio deve:
\begin{enumerate}
\item \textbf{Identificare gli errori} (Error Detection), ovvero delimitare tutte le sezioni contenenti errori nel testo, senza delimitare sezioni corrette.
\item \textbf{Correggere gli errori} (Error Correction), ovvero ripristinare il testo originale nelle sezioni individuate in precedenza
\end{enumerate}

Nelle seguenti sottosezioni sono quindi esposti alcuni dei principali approcci per letteratura, raggruppati nelle seguenti categorie di metodologie:
\begin{itemize}
\item Approcci basati su n-grams
\item Approcci basati su NMT
\item Approcci basati su BERT
\end{itemize}

\subsection{Approcci basati su n-grams}
Per discutere gli approcci basati sugli n-grams è prima necessario definire i concetti di token, tokenizzazione e n-gram. 

\paragraph{Token}
\E\ riportata la definizione fornita in \cite{tokendef}: "Un token è una stringa di caratteri contigui compresi fra due spazi, o fra uno spazio e un segno di punteggiatura. Sono token anche [numeri] interi, [numeri] reali o numeri contenenti i due punti (ore, ad esempio 2:00). Tutti gli altri simboli sono considerati essi stessi dei token, eccetto gli apostrofi e i punti di domanda attaccati ad una parola (senza spazi), che in molti casi rappresentano acronimi o citazioni."\\
Più informalmente è possibile associare il concetto di token a quello di parola nel linguaggio naturale.

\paragraph{Tokenizzazione}
Data la precedente definizione di token, per tokenizzazione si intende il dividere un testo, una frase, o più in generale una stringa nei token che la compongono. Data quindi una frase $f \in F$, tokenizzare una frase vuol dire applicare una funzione:
\begin{equation}
\textit{Tok}: F \rightarrow T
\end{equation}
dove ogni $t \in T$ è una lista $[t_1,...,t_n]$ in cui ogni $t_i$ è un token appartenente alla frase iniziale.
Più informalmente quindi, la tokenizzazione restituisce le singole parole appartenenti alla frase iniziale. Ad esempio, data la frase $f_{es}$:
\begin{center}
\textit{"Cantami, o Diva, del pelide Achille
l'ira funesta che infiniti addusse
lutti agli Achei"}
\end{center}
la sua versione tokenizzata $\textit{Tok}(f_{es})$ è:
\begin{center}
[\textit{"Cantami"},
\textit{","},
\textit{"o"},
\textit{"Diva"},
\textit{","},
\textit{"del"},
\textit{"pelide"},
\textit{"Achille"},
\textit{"l'"},
\textit{"ira"},
\textit{"funesta"},
\textit{"che"},
\textit{"infiniti"},
\textit{"addusse"},
\textit{"lutti"},
\textit{"agli"},
\textit{"Achei"}]
\end{center}

\paragraph{n-gram} Un n-gram è una sottosequenza contigua di n elementi di una data sequenza \cite{itwiki:ngram}. Gli elementi in questione possono essere fonemi, sillabe, lettere parole ecc. Nel proseguo di questo documento ogni riferimento a n-gram, salvo indicazione contraria, si riferisce a n-gram di token. Gli n-gram trovano ampio uso nel campo del NLP, dove sono usati, ad esempio, per creare modelli linguistici statistici.\\
In seguito è mostrato un esempio di scomposizione di una frase in n-gram di lunghezza 3, detti quindi 3-gram o trigrams. Dato la frase del precedente esempio $f_{es}$, e data la sua scomposizione in token $\textit{Tok}(f_{es})$, i trigrams formati sono i seguenti:
\begin{center}
\textit{["Cantami", ",", "o"]}, \textit{[",", "o", "Diva"]}, \textit{["o", "Diva", ","]}, \textit{["Diva", ",", "del"]}, \textit{[",", "del", "pelide"]}, \textit{["del", "pelide", "Achille"]}, \textit{["pelide", "Achille", "l'"]}, \textit{["Achille", "l'", "ira"]}, \textit{["l'", "ira", "funesta"]}, \textit{["ira", "funesta", "che"]}, \textit{["funesta", "che", "infiniti"]}, \textit{["che", "infiniti", "addusse"]}, \textit{["infiniti", "addusse", "lutti"]}, \textit{["addusse", "lutti", "agli"]}, \textit{["lutti", "agli", "Achei"]}
\end{center}


\paragraph{Approcci basati su n-grams}
\newcommand{\gw}{GW5}
Gli approcci descritti in questa sezione utilizzano modelli linguistici basati su n-gram per individuare e correggere gli errori. Le soluzioni proposte in questa sezione fanno entrambe uso del Google Web 1T 5-gram dataset\cite{google1t}, che da qui in poi verrà riferito come 	\gw. \gw\ è un dataset contenente n-grams in lingua inglese (da unigrams a 5-grams) associati alla loro frequenza osservata su un totale di 1 trilione di parole. Tutti gli n-grams sono stati estratti attraverso il crawling di pagine web. L'enorme scala del database e la metodologia tramite la quale è stato ottenuto comporta che sia possibile estrarre da \gw\ un amplio lessico che può essere affabilmente usato per fare error detection. Per lo stesso motivo, il dataset si presta bene anche all'applicazione in campi con terminologie di nicchia o altamente specifiche.\\
Il primo approccio trattato quello presentato in \cite{ocrG1}. L'approccio è diviso di tre fasi:
\begin{enumerate}
\item Error Detection: sono usati gli unigram in \gw\ per identificare gli errori. Ogni token all'interno del testo da correggere non presente nella lista degli unigram è considerato un errore. \E\ quindi chiaro come questo metodo riesca ad individuare (e quindi correggere) solo i non-word errors, ovvero tutti quelli errori che risultano in parole non presenti in un dato lessico. Non sono trattati da questo approccio i real-word errors, ovvero tutti quelli errori che risultano in una parola presente in un dato lessico. Si pensi ad esempio alla parola \textit{"sale"} interpretata come \textit{"sala"} da un software OCR.

\item Candidate Spelling Generation: per ogni errore si produce una lista di parole candidate per la correzione. Per fare ciò si scompone la parola errata in 2-grams a livello di carattere. Ad esempio, la parola "sangle" è scomposta in "sa", "an", "ng", "gl", "le". Per ognuna delle parole nel lessico di unigram, in seguito, si conta quante occorrenze dei 2-gram della parola da correggere sono contenute. Ad esempio, la parola "single" contiene tre occorrenze ("ng","gl","le"). Le prime 10 parole con più occorrenze dei 2-gram sono considerate i candidati per la correzione.

\item Error Correction: si considera il 5-gram terminante con la parola errata $[t_1,t_2,t_3,t_4,\textit{err}]$. Per ognuno dei candidati $c_i$ è prodotto il 5-gram $[t_1,t_2,t_3,t_4,{c_i}]$: di questi 5-gram prodotti, quello con più occorrenze all'interno di \gw\ contiene la correzione da applicare.
\end{enumerate}

Un approccio simile è esposto in \cite{ocrG2}. A differenza dell'approccio precedente, l'error detection e la generazione dei candidati non usano \gw, ma sono usate altre tecniche che mirano a correggere anche i real-word errors. L'approccio utilizzato per l'error correction invece sfrutta lo stesso principio di \cite{ocrG1}, ma con una logica leggermente più complessa. Il funzionamento è il seguente:
\begin{enumerate}
\item Error Detection e Candidate Generation: per i non-word errors GNU-Aspell\cite{atkinsongnu} è utilizzato per individuare gli errori e proporre i possibili candidati per la correzione. Per i real-word errors invece, sono usati dei confusion-set pre-definiti per individuare i possibili errori e generare i candidati. Un confusion-set non  è altro che un insieme di parole simili che possono essere confuse fra di loro, come \{they'
re, their, there\}. I candidati per un possibile errore sono quindi le parole appartenenti al suo confusion set.

\item Error Correction. Per ogni candidato si considera un intorno di 2 parole, andando così a comporre un 5-gram. Se il 5-gram in cui è presente l'errore nel testo è $[t_1,t_2,\textit{err},t_3,t_4]$, allora per il candidato $c_i$ sarà $[t_1,t_2,c_i,t_3,t_4]$. Il candidato scelto è quello il cui 5-gram compare più volte in \gw. In caso nessun 5-gram compaia nel dataset, il processo si ripete con il 3-gram $[t_2,c_i,t_3]$ e successivamente solo con l'unigram $[c_i]$, ovvero viene scelto il candidato con la maggior frequenza nel corpus.
\end{enumerate}
Gli approcci descritti fino ad ora, seppur molto efficaci contro i non-word errors, ma non correggono o sono poco efficaci contro i real-word errors e altri tipi di errori. Si pensi ad esempio a situazioni in cui token viene separato da spazi, o in cui due token sono fusi insieme. Inoltre questi approcci, richiedono un'elevata quantità di dati per funzionare efficacemente (\gw\ occupa 87GiB su disco) , il che non li rende facilmente applicabili.

% Potrei aggiungere cose anche da https://aclanthology.org/H05-1109.pdf
% o da https://ieeexplore.ieee.org/stamp/stamp.jsp?tp=&arnumber=8978061

\subsection{Approcci basati su NMT}
%Neural machine translation is a recently proposed approach to machine translation. Unlike the traditional statistical machine translation, the neural machine
%translation aims at building a single neural network that can be jointly tuned to
%maximize the translation performance.
\paragraph{Neural Machine Translation} Il Neural Machine Translation (NMT) è un approccio al machine translation che consiste nella costruzione di un solo neural network che può essere messo a punto singolarmente per massimizzare le performance di traduzione\cite{nmtdef}. I modelli più recenti nel campo del NMT sfruttano la cosiddetta architettura encoder-decoder (\autoref{fig:art_encdec}). Tali modelli sono anche comunemente riferiti come modelli sequence-to-sequence (seq2seq), in quanto trasformano una sequenza di caratteri o lettere appartenente a un dominio (ad esempio quello dell frasi in lingua italiana) ad una sequenza appartenente ad un altro dominio (ad esempio quello delle frasi in lingua inglese). In questi modelli, la frase iniziale passa attraverso l'encoder, che ne restituisce una rappresentazione sotto forma di una lista di valori numerici. Successivamente, questi valori sono dati in input al decoder, che produce una frase con lo stesso significato, ma con il vocabolario e la grammatica della lingua target. Encoder e decoder possono essere implementati con diverse architetture, ma, data la natura dei dati trattati (dati sequenziali, come frasi), sono spesso implementati con dei Recurrent Neural Network (RNN).


\begin{figure}[H]
\centering
\includegraphics[width=\textwidth]{immagini/stato_arte/encoder_decoder}
\caption{Schema semplificato di un'architettura encoder-decoder che traduce da italiano a inglese}
\label{fig:art_encdec}
\end{figure}

L'idea che gli approcci in questa sezione propongono è quindi quella di trattare il problema della correzione degli errori come un problema di traduzione di sequenze: dato il dominio $E$ delle sequenze contenenti errori, e il dominio $C$ delle sequenze corrette (detto anche Ground Truth, abbreviato GT), il modello seq2seq che effettua la traduzione può essere visto come la seguente funzione:
\begin{equation}
M: E \rightarrow C
\end{equation}
Allo stesso modo, ogni modello necessita prima di essere allenato con corpus parallelo dove ad ogni sequenza $e \in E$ corrisponde un sequenza corretta $c \in C$.

\paragraph{Modelli Word Based e Character based} Come si è visto, i modelli seq2seq trasformano una sequenza in un'altra sequenza. In base al tipo di elementi che compongono tali sequenze, è possibile descrivere due tipologie di modelli:
\begin{itemize}
\item Modelli Word Based (da qui in poi riferiti come WB): in questo tipo modelli un elemento corrisponde ad una parola, o meglio, ad un token. Questo tipo di approcci richiede quindi che tutte le sequenze in $E$ e in $C$ siano tokenizzate. Inoltre, prima dell'allenamento, è necessario estrarre i dizionari della lingua d'origine e della lingua target dai corpora utilizzati.
\item Modelli Character Based (da qui in poi riferiti come CB): in questo tipo di modelli un elemento corrisponde ad un carattere. Prima del training, è necessario estrarre il set di caratteri della lingua d'origine e della lingua target dai corpora utilizzati.
\end{itemize}
Da un'analisi della letteratura, si evince come i modelli CB siano generalmente preferiti per il task dell'OCR Post-Processing\cite{mokhtar2018ocr,hamalainen2019paft,nguyen2020neural,
nastase2018correction,duong2020unsupervised,amrhein2018supervised}.
In \cite{mokhtar2018ocr} viene proposto un confronto sullo stesso corpus fra due architetture molto simili, la prima WB e la seconda CB. Nello studio, gli autori concludono come l'architettura character based produca risultati significativamente migliori. I modelli CB infatti riescono a correggere gli errori a livello di carattere, e ciò comporta i seguenti vantaggi:
\begin{itemize}
\item A differenza dei modelli WB, i modelli CB possono correggere anche errori in parole non presenti nel dizionario estratto.
\item Rispetto ai modelli WB, sono richiesti meno dati durante la fase di allenamento.
\end{itemize}


\paragraph{Approcci basati su NMT}
Il primo approccio presentato è quello descritto in \cite{nguyen2020neural}. Nella soluzione presentata gli autori utilizzano il tool OpenNMT\cite{klein2017opennmt} per allenare un modello CB. Il modello è allenato sia con coppie di sequenze identiche (quindi senza errori) che con sequenze dove la frase acquisita tramite OCR e la GT differiscono. Per queste ultime è adottata la seguente strategia di data augmentation: per ogni errore, si costruiscono cinque 5-gram con il token errato e i suoi token vicini facendo scorrere una finestra sopra la sezione di testo contenente il token errato. In questo modo si producono 5 sequenze diverse che possono essere usate per il training, evitando di un avere un modello con troppo bias verso le sequenze senza errori. L'approccio per la correzione si divide dunque in due fasi: un prima fase di error detection utilizza un modello BERT (si rimanda il lettore alla \autoref{sec:art_bert}) messo a punto per la classificazione per individuare gli errori nelle sequenze; una seconda fase di error correction invece utilizza il modello di NMT allenato precedentemente per correggere gli errori.\\
In \cite{amrhein2018supervised} si utilizza una metodologia simile per il training del modello, sviluppato con il framework Nematus\cite{sennrich2017nematus}. Questo approccio utilizza una tecnica chiamata "factored NMT", utilizzata anche in \cite{nguyen2020neural}, che permette di aggiungere informazioni strutturate alle sequenze in input. Ad esempio, è possibile aggiungere ad ogni carattere della sequenza informazioni sull'identificativo e l'anno del corpus di provenienza. Ciò permette di usare più corpora eterogenei per l'allenamento di un solo modello, aumentando così i dati a disposizione.\\
\cite{nastase2018correction} propone invece un approccio per correggere unicamente i word segmentation errors (si rimanda il lettore alla \autoref{sec:met_introduzione}), ovvero errori introdotti da spazi spuri che influiscono sulla segmentazione del testo. La metodologia proposta allena un modello su coppie di sequenze così composte: la sequenza target è una sequenza corretta senza errori; la sequenza di input invece è la sequenza target senza spaziature. In questo modo, il modello è allenato per inserire correttamente gli spazi e può essere usato per correggere i word segmentation error.\\
In \cite{hamalainen2019paft} si propone invece un approccio per il post processing e l'allenamento di un modello di NMT senza disporre di un corpus parallelo. L'approccio propone di allenare modello Word2Vec direttamente sul corpus contenente gli errori introdotti dal software di OCR. Dopodiché, per ogni parola estratta corretta (la correttezza è verificata attraverso un dizionario) si usa il modello Word2Vec per trovare le parole semanticamente più simili. Di queste, quelle con distanza di Levenshtein minore o uguale di 3 sono considerate versioni errate della parola corretta. Successivamente, le coppie di parole corrette e parole versione errate sono usate per allenare il modello di NMT, che è poi utilizzato per la correzione. Dato che il risultante modello è allenato per la correzione di singole parole, gli errori di segmentazione non vengono corretti. Inoltre, si rende necessaria una fase di error detection basata su un dizionario. \\
L'approccio proposto è ulteriormente sviluppato in \cite{duong2020unsupervised}. Gli autori procedono sempre con la creazione di un corpus parallelo per l'allenamento di un modello di NMT, ma con una diversa strategia. In questo caso si parte da sequenze senza errori, che fungono da input per l'allenamento, e si creano sequenze target mediante l'introduzione di errori in maniera controllata. Inoltre, in questo caso, le sequenze usate per l'allenamento sono degli 5-gram centrati sulla parola errata, con lo scopo di fornire un maggior contesto per migliorare la correzione.


\subsection{Approcci basati su BERT}
\label{sec:art_bert}
%https://towardsdatascience.com/bert-explained-state-of-the-art-language-model-for-nlp-f8b21a9b6270
\paragraph{BERT}
BERT\cite{devlin2018bert} (Bidirectional Encoder Representation from Transformers) è un'architettura per la costruzione di modelli linguistici sviluppata da Google sulla base dei modelli transformers. Una delle differenze fra BERT e un transformer base consiste nella sola presenza di un encoder. Infatti, essendo BERT progettato solo per la modellazione linguistica (e non per la trasformazione di sequenze), non si rende necessaria la presenza di un decoder. Inoltre, a differenza dei modelli direzionali, in cui l'encoder legge il testo in input da sinistra verso desta o viceversa, l'encoder di BERT legge l'intera sequenza all'unisono. Ciò consente al modello di comprendere il contesto di una parola in base al testo a destra e sinistra di quest'ultima. I modelli BERT sono pre-allenati minimizzando la perdita combinata per i task di Masked Language Modelling (MLM) (\autoref{sec:met_BERT_MLM}) e Next Sentence Prediction (NSP). Dopo la fase di pre-allenamento è possibile eseguire il fine-tuning del modello per uno specifico task di NLP, come Classification, Question Answering e Named Entity Recognition (NER).

\paragraph{Approcci basati su BERT}
In \cite{nguyen2020neural}  gli autori eseguono il fine tuning di un modello BERT per eseguire error detection in un dataset in lingua inglese. Ciò avviene dividendo in un primo momento la frase in input in sub-token tramite WordPiece. Il modello BERT classifica quindi i subtoken come validi o non validi: i token che contengono subtoken non validi sono segnalati quindi come errore.\\
In \cite{OCRMaskFilling} gli autori usano una combinazione di BERT e FastText per generare possibili candidati per la correzione di errori introdotti da software OCR in testi in lingua inglese. Limitandosi ai 50 candidati più probabili prodotti con entrambi i metodi, la giusta correzione è presente fra i candidati nel 71\% dei casi. Non è tuttavia proposto un metodo per scegliere il candidato corretto.\\
In \cite{zhang2020spelling} gli autori combinano un neural network di classificazione binaria per l'error detection e un'ulteriore neural network basato su BERT per l'error correction. I token (che in questo caso sono caratteri, dato che l'approccio lavora su testi in lingua cinese)  marcati come errore sono mascherati dal correction network, che propone e sceglie successivamente le soluzioni con più probabilità di essere corrette. I due network sono allenati congiuntamente con un approccio end-to-end, con la loss function che è data da una combinazione lineare delle loss function di correction e detection network.
























\chapter{Dataset e Perturbazione}
\label{sec:dataset}
In questo capitolo è descritto il processo di creazione del dataset usato per valutare il sistema di correzione sviluppato.\\
Nella \autoref{dst:intro} sono discussi gli obiettivi e le caratteristiche che il dataset deve avere. Nella \autoref{dst:creazione} sono delineate la metodologia di creazione e la struttura del dataset. Nella \autoref{dst:perturbazione} è descritto il processo introduzione di errori all'interno delle frasi per simularne l'acquisizione tramite OCR, detto perturbazione. Nella \autoref{dst:configurazione} infine, sono discussi i parametri e la configurazione del dataset.
\section{Introduzione}
\label{dst:intro}
Lo sviluppo di un sistema di OCR post-processing non può prescindere dall'esecuzione di test per valutare, migliorare e confrontare le soluzioni adottate. Ciò rende necessario l'utilizzo di un dataset sul quale poter eseguire tutte le batterie di test necessarie. Tale dataset deve inoltre essere strutturato in modo tale da permettere di calcolare facilmente le metriche descritte nella \autoref{sec:test_metriche}. Se infatti lo scopo di un sistema di OCR post-processing è quello di correggere gli errori introdotti dai software di OCR, è necessario conoscere la posizione degli errori per controllare se siano stati corretti o meno.\\
La soluzione più semplice a questo problema è quella di disporre, per ogni frammento del dataset, sia del testo acquisito tramite OCR (e quindi contenente errori), sia del testo originale, detto anche ground truth. Non è però semplice trovare dataset di questo tipo, specialmente se in lingua italiana. Spesso, infatti, è necessario che la ground truth sia acquisita manualmente con un grosso dispendio di tempo e risorse: a causa di ciò, questo approccio non è stato considerato percorribile ai fini della tesi.\\
Si è invece deciso di adottare un approccio differente, che consiste nell'introduzione artificiale di errori all'interno di testo nativamente digitale (e quindi senza errori) per simularne l'acquisizione tramite OCR. I vantaggi del processo appena descritto, detto "perturbazione", sono riassumibili nei seguenti punti:
\begin{itemize}
 \item Produrre il dataset, una volta definita la funzione che esegue la perturbazione, è molto meno oneroso in termini di tempo e risorse. L'unico requisito è quello di procurarsi una collezione sufficientemente ampia di testo in formato digitale. Ciò è relativamente semplice, e può essere fatto, ad esempio, tramite web scraping. 
 \item \E\ possibile definire arbitrariamente l'intensità e la tipologia degli errori nel testo tramite l'uso di diverse funzioni di perturbazione. Ciò permette, dato un singolo testo di partenza, di ottenere testi con diverse intensità e tipologie di errori. In questo modo si rende possibile valutare le performance del sistema di OCR post-processing al variare delle condizioni del testo dato in input.
\end{itemize}
Il principale rischio che si corre utilizzando l'approccio appena descritto risiede nel fatto che il testo perturbato potrebbe non rispecchiare fedelmente gli errori presenti in testo acquisito realmente tramite software di OCR.
\section{Creazione del dataset}
\label{dst:creazione}
Il dataset di partenza è una collezione di 15073 documenti in formato JSON, ottenuti mediante il web scraping del sito ufficiale del vaticano \url{www.vatican.va}. I documenti contengono preghiere, lettere, discorsi, encicliche ecc. editi da figure ecclesiastiche dal 1439 al 2021.\\
Ogni documento è caratterizzato dalla struttura descritta in \autoref{tab:dst_docstrut}.


\begin{table}[H]
\centering
\begin{tabular}{ll}
\textbf{Nome campo} & \textbf{Contenuto} \\ \hline
title & Titolo del documento \\
description & Insieme di keyword del documento\\
author & Autore del testo nel documento \\
creator & Coincide con author \\
language & Codice della lingua del documento (es. it, fr) \\
date & Data di redazione del documento \\
url & Url da cui è stato il documento \\
class & Tipologia di documento (es. discorso, preghiera...)\\
text & Testo completo estratto dal documento (titolo compreso)\\
url\_references & Link al feed RSS del sito del vaticano \\
text\_references & Posizione della pagina nella gerarchia del sito\\
italic & Testo estratto in corsivo nella pagina \\
paragraphs & Testo estratto diviso in paragrafi contenenti id e testo
\end{tabular}
\caption{Struttura di un documento nel dataset}
\label{tab:dst_docstrut}
\end{table}

Dato il dataset descritto, l'obiettivo è quello di trasformalo in un formato consono al testing. Nello specifico, è necessario frammentare il testo contenuto nei documenti in frasi di lunghezza minima e massima predeterminate. Le ragioni di questa decisione sono approfondite nel \autoref{sec:test}, ma, in breve, spezzare il testo in frammenti più piccoli facilita il processo di allineamento delle frasi e quindi la valutazione delle correzioni. Per ognuno dei frammenti si vogliono poi produrre più versioni perturbate con diverse funzioni di perturbazione. Infine, è necessario che le frasi siano accompagnate da dei metadati che consentano la ricomposizione del testo originale mediante il riordinamento dei frammenti.\\
La creazione del dataset per il testing, descritta nelle prossime sezioni, segue dunque le seguenti fasi:
\begin{enumerate}
\item Filtraggio lingue
\item Estrazione
\item Frammentazione
\item Filtraggio paragrafi
\item Perturbazione
\item Riduzione
\end{enumerate}

\paragraph{Filtraggio lingue}
In questa fase vengono filtrati tutti i documenti non presenti fra lingue scelte. Dato l'insieme dei documenti $D$ e l'insieme delle lingue consentite $L$, l'insieme filtrato dei documenti $D_{fl}$ si ottiene come segue:
\begin{equation}
D_{fl} = \{d\ \forall d \in D\ |\ \textit{language}_d \in L  \} 
\end{equation}
dove, dato un documento $d \in D$, $\textit{language}_d$ è la lingua in cui è redatto. Le lingue scelte per la creazione del dataset sono elencate nella \autoref{dst:configurazione}.

\paragraph{Estrazione}
In questa fase sono scartati i campi superflui ai fini del testing, e sono mantenuti solo i paragrafi assieme ad alcuni metadati utili a ricomporre il testo originale in seguito. \E\ possibile formalizzare la prima parte di questa fase come segue:
\begin{equation}
D_{es1} = \{ meta({paragraphs}_d)\ \forall d \in D  \}
\end{equation}
dove la funzione $meta$ aggiunge ai paragrafi in ${paragraphs}_d$ l'id del documento di provenienza. Si ottiene quindi un insieme di insiemi di paragrafi, dove ogni paragrafo è strutturato come in \autoref{tab:dst_parstrut}.

\begin{table}[H]
\centering
\begin{tabular}{ll}
\textbf{Nome campo} & \textbf{Contenuto} \\ \hline
text  & Testo contenuto nel paragrafo\\
parId & Codice del paragrafo all'interno del documento \\
docId & Codice del documento di provenienza del paragrafo.
\end{tabular}
\caption{Struttura di un paragrafo}
\label{tab:dst_parstrut}
\end{table}

La fase di estrazione è completata appiattendo le liste di paragrafi nell'insieme $D_{es1}$ 
\begin{equation}
D_{es} = \bigcup\limits_{p \in D_{es1}} p
\end{equation}
$D_{es}$ contiene quindi tutti i paragrafi del dataset iniziale strutturati come mostrato in \autoref{tab:dst_parstrut}.



\paragraph{Frammentazione}
Lo scopo di questa fase è quello di scomporre i paragrafi estratti nella fase precedente in frammenti più brevi, detti frasi, con le seguenti caratteristiche:
\newcommand{\lmin}{$l_{min}$}
\newcommand{\lmax}{$l_{max}$}
\begin{itemize}
\item Ogni frase è disgiunta dalla altre frasi estratte dal medesimo paragrafo.
\item Ogni frase è compresa fra una lunghezza minima \lmin\ e una lunhezza massima \lmax\ (\autoref{dst:configurazione}).
\end{itemize}

Dato quindi il testo di un paragrafo $text_p$, la frammentazione avviene secondo la seguente procedura:

\begin{enumerate}
\item Se la lunghezza di $text_p$ è minore di \lmin, non si procede e si scarta la stringa.
\item Si divide $text_p$ su un segno di punteggiatura fra i seguenti: "?", "!", ";", ":", ".", ",". Se sono disponibili più opzioni (ovvero più punti in cui è possibile dividere su segni di punteggiatura) si divide nel punto minore o uguale a \lmax\ che più si avvicina al valore di \lmax. La parte a sinistra del punto di divisione è una frase estratta. La parte a destra invece viene riutilizza come input della procedura di frammentazione, tornando al punto 1. Se non fosse possibile eseguire alcuna divisione, si passa al punto successivo.
\item Si divide $text_p$ su uno spazio. Se sono disponibili più opzioni (ovvero più punti in cui è possibile dividere su uno spazio) si divide nel punto minore o uguale a \lmax\ che più si avvicina al valore di \lmax. La parte a sinistra del punto di divisione è una frase estratta. La parte a destra invece viene riutilizza come input della procedura di frammentazione, tornando al punto 1. Se non fosse possibile eseguire alcuna divisione, si passa al punto successivo.
\item Se la lunghezza di $text_p$ è minore di \lmax, $text_p$ è aggiunto alle frasi estratte. Altrimenti, si divide $text_p$ in posizione \lmax. La parte a sinistra del punto di divisione è una frase estratta. La parte a destra invece viene riutilizza come input della procedura di frammentazione, tornando al punto 1. 
\end{enumerate}

La procedura appena descritta fa in modo che i frammenti approssimino il più possibile \lmax, evitando il più possibile introdurre errori. Spezzare un paragrafo nel mezzo di una parola, ad esempio, andrebbe a creare due frammenti contenenti degli errori (uno sulla parola finale del primo, l'altro sulla parola iniziale del secondo.\\
Durante la frammentazione ogni frammento ritiene i metadati relativi al documento e al paragrafo di appartenenza. In più, si aggiunge un ulteriore campo per tenere traccia della posizione del frammento all'interno del paragrafo di appartenenza. In questo modo, è possibile ricomporre un documento in secondo momento. Ogni frase prodotta rispetta il seguente schema:
\begin{table}[H]
\centering
\begin{tabular}{ll}
\textbf{Nome campo} & \textbf{Contenuto} \\ \hline
text  & Testo contenuto nella frase\\
parId & Codice del paragrafo di provenienza \\
docId & Codice del documento di provenienza.\\
parPos & Posizione della frase all'interno del paragrafo di provenienza\\
\end{tabular}
\caption{Struttura di una frase}
\label{tab:dst_frasestrut}
\end{table}

La frase di frammentazione si può quindi formalizzare in due step. Nel primo step, ogni paragrafo viene frammentato nella lista delle sue frasi:
\begin{equation}
D_{fr1} = \{ extr(p)\ \forall p \in D_{es}  \}
\end{equation}
dove $extr$ è la funzione che estrae le frasi secondo la procedura descritta precedentemente, producendo un insieme di frasi strutturate come mostrato in \autoref{tab:dst_frasestrut}. Nel secondo step si appiattisce l'insieme di insiemi che si è andato a creare, per ottenere insieme di frasi:
\begin{equation}
D_{fr} = \bigcup\limits_{f \in D_{fr1}} f
\end{equation}

\paragraph{Filtraggio paragrafi}
Lo scopo di questa fase è quello di rimuovere dal dataset tutti le frasi che appartengono al primo o all'ultimo paragrafo di un documento. Ciò si rende necessario perchè questi paragrafi spesso contengono intestazioni o piè di pagina ripetitivi andrebbero a sporcare il dataset. Ad esempio, l'ultimo paragrafo contiene spesso una frase simile  a \textit{"© Copyright 2004 - Libreria Editrice Vaticana"}, mentre il primo spesso contiene solamente una data.\\
\E\ definita la seguente funzione $\textit{maxPar}$ che, data una frase $f \in D_{fr}$, indica se $f$ appartiene all'ultimo paragrafo del documento da cui è stata estratta:
\begin{equation}
  maxPar(f) =
    \begin{cases}
      1 & \text{se $f$ appartiene all'ultimo paragrafo di $docId_f$}\\
      0 & \text{altrimenti}
    \end{cases}       
\end{equation}
L'operazione di filtraggio dei paragrafi può quindi essere formalizzata come segue:
\begin{equation}
D_{fp} = \{ f \ \forall f \in D_{fr}\ |\  parId_{f} \neq 0 \wedge maxPar(f) = 0     \}
\end{equation}

\paragraph{Perturbazione}
Durante la fase di perturbazione, per ogni frase del dataset sono generate le sue versioni perturbate. Ogni versione perturbata di una frase viene generata da una diversa funzione di perturbazione applicata alla frase stessa. Il funzionamento delle funzioni di perturbazione è spiegato nella \autoref{dst:perturbazione}, mentre le esatte funzioni per la creazione del dataset sono elencate nella \autoref{dst:configurazione}.\\
Viene chiamata $F_p$ la lista $[p_1,...,p_n]$ delle funzioni di perturbazione utilizzate, in cui ogni $p_i$ è identificata da un codice. Data una stringa di testo $s$, la sua versione perturbata dalla funzione $p_i$ corrisponde a $p_i(s)$.\\
Data quindi una frase $f \in D_{fr}$ strutturata come in \autoref{tab:dst_frasestrut}, e lista delle funzioni di perturbazione $F_p$, si definisce come $P_{list}$ l'insieme delle frasi perturbate $[p_1(text_f),...,p_n(text_f)]$ la cui rappresentazione è schematizzata in \autoref{tab:dst_frasepert}.

\begin{table}[H]
\centering
\begin{tabular}{ll}
\textbf{Codice funzione perturbazione} & \textbf{Frase perturbata} \\ \hline
$nome\_func_1$ & $ p_1(f)$\\
$nome\_func_2$ & $ p_2(f)$\\
... &  ...\\
$nome\_func_n$ & $ p_n(f)$\\
\end{tabular}
\caption{Struttura delle frasi perturbate}
\label{tab:dst_frasepert}
\end{table}

Si definisce inoltre la funzione $applyPert$ che, data una frase $f \in D_{fr}$, produce un sample aggiungendo a $f$ un campo con le sue versioni perturbate della frase originale. Un sample è strutturato come in \autoref{tab:dst_samplestrut}.

\begin{table}[H]
\centering
\begin{tabular}{ll}
\textbf{Nome campo} & \textbf{Contenuto} \\ \hline
text  & Testo contenuto nella frase non perturbata\\
parId & Codice del paragrafo di provenienza \\
docId & Codice del documento di provenienza.\\
parPos & Posizione della frase all'interno del paragrafo di provenienza\\
perturbed & Frasi perturbate ($[p_1(text_f),...,p_n(text_f)]$)
\end{tabular}
\caption{Struttura di una frase}
\label{tab:dst_samplestrut}
\end{table}


La fase di perturbazione può quindi essere formalizzata come segue:
\begin{equation}
D_{pe} = \{applyPert(f)\ \forall f \in D_{fp}\}
\end{equation}

\paragraph{Riduzione}
Lo scopo della fase di riduzione è quello di creare una versione ridotta del dataset, ovvero di ridurne il numero di elementi. Ciò è reso necessario dal fatto che il testing può essere significativamente oneroso in termini di tempo e risorse, specialmente su dataset troppo ampi.\\
Data quindi la dimensione del dataset ridotto $n$ (i cui valori sono specificati nella \autoref{dst:configurazione}), il dataset $D_{re}$ è un sottoinsieme di $n$ elementi estratti casualmente da $D_{pe}$:
\begin{equation}
D_{re} \subset D_{pe} \wedge |D_{re}| = n
\end{equation}
Dopo aver eseguito la fase di riduzione, la costruzione del dataset può dirsi completata.


\section{Perturbazione}
\label{dst:perturbazione}
Data una stringa $s$, lo scopo del processo di perturbazione è quello produrre una stringa $s\prime$ nella quale sono introdotti degli errori per simulare l'acquisizione del testo tramite OCR. Per permettere l'introduzione controllata e a diverse intensità degli errori è stato sviluppato un sistema modulare e componibile, che sarà illustrato in questa sezione.

\subsection{Moduli di perturbazione}
\label{dst:modpert}
La componente base del sistema di perturbazione è il modulo di perturbazione. Un modulo di perturbazione non è altro che una funzione che modella l'inserimento di una specifica tipologia di errore. Ciò significa che per ogni tipologia di errore modellata si rende necessaria la creazione di un modulo di perturbazione. Ad esempio, è possibile definire un modulo di perturbazione che modelli la sostituzione di alcuni caratteri, o un modulo che inserisca della punteggiatura spuria. \\
Tutti i moduli di perturbazione, data una lista ordinata di token, introducono errori mediante la modifica, la divisione, l'unione o la cancellazione di uno o più token. \\
Data una lista di token $list_t = [t_1,...,t_n]$,
Il funzionamento di un modulo di perturbazione si può dividere nelle seguenti tre fasi:

\begin{enumerate}
	\item \textbf{Raggruppamento}: I token sono raggruppati in gruppi consecutivi e disgiunti di $k$ elementi, dove $k$ varia a seconda della tipologia di modulo di perturbazione. \\
La fase di raggruppamento è quindi definita come segue:
\begin{equation}
group: [t_1,...,t_n] \rightarrow [g_1,...,g_q]
\end{equation}
dove ogni $g_i$ è un gruppo di token $[t_1,..., t_k  ]$ e $q = \lceil n/k \rceil$.\\
	
Ad esempio, data la lista di token \textit{['Nel', 'mezzo', 'del', 'cammin', 'di', 'nostra', 'vita']} e $k = 2$, i token sono raggruppati nel seguente modo: \textit{['Nel', 'mezzo'], ['del', 'cammin'], ['di', 'nostra'], ['vita']}.
	
	\item \textbf{Perturbazione}: Ogni tipologia di modulo di perturbazione è caratterizzata da una specifica funzione di perturbazione $f_{pert}$
\begin{equation}
f_{pert}: [t_1,..., t_k  ] \rightarrow [t'_1,..., t'_j]
\end{equation}
che introduce errori mediante la modifica, la divisione, l'unione o la cancellazione dei token in un determinato gruppo, con una probabilità $p$ definita per ogni modulo. Un qualsiasi gruppo di token $g \in [g_1,...,g_q]$ viene perturbato o meno secondo la seguente funzione:
\begin{equation}
f_{map}(g) =
    \begin{cases}
      f_{pert}(g) & \text{con probabilità $p$}\\
      g & \text{con probabilità $1-p$}
    \end{cases}  
\end{equation}
\E\ quindi possibile definire la fase di perturbazione come:
\begin{equation}
pert: [g_1,...,g_q] \rightarrow [g\prime_1,...,g\prime_q]
\end{equation}
dove $g\prime_i = f_{map}(g_i)\ \forall i \in [1..q]$.


Ad esempio, si supponga di avere un modulo in cui $f_{pert}$ sia definita come:
\begin{equation}
f_{ex}([t_1,t_2]) = [t_1 \frown t_2]
\end{equation}
La funzione appena definita concatena i due token all'interno di gruppo, restituendo un gruppo formato da un solo token. Si consideri l'esempio nel punto precedente, e si supponga che l'unico gruppo affetto da perturbazione sia il secondo. Si ha quindi che $f_{pert}(\textit{['del', 'cammin']}) = \textit{['delcammin']}$, e quindi alla fine di questa fase si ottiene: \textit{['Nel', 'mezzo'], ['delcammin'], ['di', 'nostra'], ['vita']}.

\item \textbf{Appiattimento}: la fase di appiattimento è l'opposto di quanto avviene durante il raggruppamento. I gruppi di token sono raggruppati nuovamente in un'unica lista:
\begin{equation}
flat: [g\prime1,...,g\prime_q] \rightarrow [t_1,...,t_m]
\end{equation}
dove $m$ potrebbe coincidere o meno con $n$ a seconda del tipo di perturbazione avvenuta.\\
Riprendendo l'esempio precedente, il risultato di questa fase è: \textit{['Nel', 'mezzo', 'delcammin', 'di', 'nostra', 'vita']}
\end{enumerate}

Come si può notare, un modulo riceve in input una lista di token e fornisce in output un'altra lista di token. Ciò rende possibile concatenare più moduli, per introdurre più di un tipo di errore all'interno di una sequenza di token. Questa eventualità è descritta nella \autoref{sec:dst:pipeline} e nella \autoref{sec:dst:superpipeline}.\\
Quanto descritto finora non basta a soddisfare lo scopo della perturbazione, che è quello di trasformare una stringa in una nuova stringa contenente degli errori. Tale funzione è resa possibile dai moduli speciali, che sono approfonditi nella \autoref{sec:dst_modspec}.\\
Dal funzionamento dei moduli di perturbazione si evince come la funzione $f_{pert}$ caratterizzi il modulo modellando il tipo di errore che esso va ad introdurre. Sono quindi stati creati i seguenti moduli, ognuno dei quali è caratterizzato da una diversa funzione di perturbazione:

\newcommand{\mss}{$M_{ss}$}
\newcommand{\mps}{$M_{ps}$}
\newcommand{\mpi}{$M_{pi}$}
\newcommand{\mhm}{$M_{hm}$}
\newcommand{\mcr}{$M_{cr}$}

\begin{itemize}
\item Modulo di space splitting (\mss)
\item Modulo di punctuation splitting (\mps)
\item Modulo di punctuation insertion (\mpi)
\item Modulo di hyphen merging (\mhm)
\item Modulo di characters replacement (\mcr)
\end{itemize}

Ognuno di questi moduli è stato creato per modellare un particolare tipo di errore. Le tipologie di errore modellate sono state decise in base all'analisi di documenti il cui testo è stato acquisito tramite OCR. Più nello specifico, si tratta delle trascrizioni dei dibattiti parlamentari della prima e della seconda legislatura della repubblica italiana.

\subsubsection{Modulo di space splitting}
Il modulo di space splitting modella l'errore in cui le lettere di una parola sono intramezzate da degli spazi. Siccome la perturbazione avviene su un singolo token, i token in input sono raggruppati in gruppi di un token ($k = 1$). Si definisce dunque la funzione di perturbazione del modulo:
\begin{equation}
f_{pert\_split}([t_1]) = [f_{split}(t_1)]
\end{equation}
in cui $f_{split}$ è la funzione che aggiunge uno spazio dopo ogni carattere di un token.
\paragraph{Esempi}
\begin{equation}
f_{pert\_split}(["cammin"]) = ["c\ a\ m\ m\ i\ n\ "]
\end{equation}
\begin{equation}
f_{pert\_split}(["nostra"]) = ["n\ o\ s\ t\ r\ a\ "]
\end{equation}

\subsubsection{Modulo di punctuation splitting}
Il modulo di punctuation splitting modella l'errore in cui una o più occorrenze di un segno di punteggiatura intramezzano i caratteri di una parola.
Siccome la perturbazione avviene su un singolo token, i token in input sono raggruppati in gruppi di un token ($k = 1$). Si definisce dunque la funzione di perturbazione del modulo:
\begin{equation}
f_{pert\_punctsplit}([t_1]) = [f_{punctsplit}(t_1,punct)]
\end{equation}
Assumendo che:
\begin{itemize}
\item $punct$ sia il segno di punteggiatura che è intramezzato agli spazi,
\item un token $t$ sia una sequenza di caratteri $[c_1,...,c_n]$, ovvero $|t| = n$,
\item sia definito un numero casuale $x \in \mathbb{N}\ |\ 1 \leq x \leq n$
\end{itemize}
è possibile definire $f_{punctsplit}$ come:
\begin{equation}
f_{punctsplit}(t,punct) = 
\begin{cases}
	[c1,...,c_x] \frown\ punct \frown f_{punctsplit}([c_{x+1},...,c_n],punct)&\text{se $x < |t|$}\\
	t	&\text{se $x \geqslant |t|$}
\end{cases}
\end{equation}
Informalmente, la funzione aggiunge un carattere $punct$ ogni $x$ caratteri all'interno di $t$.

\paragraph{Esempi}
\begin{equation}
f_{punctsplit}(\textit{"cammin"},\textit{","}) = \textit{"ca,mm,in"}\ (x = 2)
\end{equation}
\begin{equation}
f_{punctsplit}(\textit{"mezzo"},\textit{","}) = \textit{"m,e,z,z,o"}\ (x = 1)
\end{equation}


\subsubsection{Modulo di punctuation insertion}
Il modulo di punctuation insertion modella l'errore in cui un segno di punteggiatura è aggiunto fra due token, senza quindi dividere in due un singolo token.
Siccome la perturbazione avviene su un singolo token, i token in input sono raggruppati in gruppi di un token ($k = 1$). Si definisce dunque la funzione di perturbazione del modulo:
\begin{equation}
f_{pert\_insertion}([t_1]) = [t_1, punct]
\end{equation}
dove $punct$ è il segno di punteggiatura da aggiungere.
\paragraph{Esempi} Data la sequenza di gruppi di token \textit{\underline{['Nel']}, ['mezzo'], ['del'], \underline{['cammin']}, ['di'], ['nostra'], ['vita']}, dove i gruppi sottolineati sono quelli da perturbare, e dato $punct = "."$, si ha che:
\begin{equation}
f_{pert\_insertion}(\textit{["Nel"]},\textit{"."}) = \textit{["Nel","."]}\ 
\end{equation}
\begin{equation}
f_{pert\_insertion}(\textit{["cammin"]},\textit{"."}) = \textit{["cammin","."]}\ 
\end{equation}
Quindi, applicata la fase di appiattimento, la sequenza finale risulta: \textit{['Nel'], ['.'], ['mezzo'], ['del'], ['cammin'], ['.'], ['di'], ['nostra'], ['vita']}.

\subsubsection{Modulo di hyphen merging}
Il modulo di hypen merging modella l'errore in cui due parole consecutive vengono unite da un trattino. Dovendo unire due token tra di loro, la perturbazione avviene in gruppi di due token, quindi si ha $k = 2$. Si definisce dunque la funzione di perturbazione del modulo:
\begin{equation}
f_{hyphen}([t_1,t_2]) = [t_1\frown \textit{"-"} \frown t_2]
\end{equation}

\paragraph{Esempio} Data la sequenza di gruppi di 2 token \textit{['Nel', 'mezzo'], ['del', 'cammin'], \underline{['di', 'nostra']}, ['vita']}, dove il gruppo sottolineati sono quello da perturbare, si ha che:
\begin{equation}
f_{hyphen}(\textit{["di", "nostra"]}) = \textit{["di-nostra"]}
\end{equation}
Quindi, applicata la fase di appiattimento, la sequenza finale risulta: \textit{['Nel', 'mezzo'], ['del', 'cammin'], {['di-nostra']}, ['vita']}.

\subsubsection{Modulo di characters replacement}
Il modulo di characters replacement modella l'errore in un cui avviene la cancellazione, l'aggiunta o la sostituzione di uno o più caratteri all'interno di un token. Siccome la perturbazione avviene su un singolo token, i token in input sono raggruppati in gruppi di un token ($k = 1$). Si definisce dunque la funzione di perturbazione del modulo:
\begin{equation}
f_{charrepl}([t_1]) = [f_{alternative}(t_1)]
\end{equation}
Per ogni token $t$ che viene perturbato, esiste un insieme di 5 versioni errate di $t$, detto $A_t = [a_{t1},...,a_{t5}]$, generato sulla base di una distribuzione di errori preesistente. La funzione $f_{alternative}$ rimpiazza $t$ con una delle alternative in $A_t$:
\begin{equation}
f_{alternative}(t) = a_{ti}
\end{equation}
dove $a_{ti} \in A_t$ e $i$ è un numero intero casuale fra 1 e 5.

\paragraph{Esempio} Sia dato il token \textit{"cammin"} e si supponga l'insieme delle versioni errate del token essere $A_{cammin} = [\textit{'cmmin', 'camniin', 'cdmmn', 'camin', 'cammiii'}]$. Si sceglie quindi in modo casuale un token dall'insieme $A_{cammin}$:
\begin{equation}
f_{alternative}(\textit{"cammin"}) = a_{cammin\_2} ="camniin"
\end{equation}


\subsection{Moduli speciali}
\label{sec:dst_modspec}
Come si è visto nella sottosezione precedente, i moduli di perturbazione lavorano su sequenze di token, nelle quali introducono degli errori. \E\ però necessario, come accennato all'inizio della \autoref{dst:perturbazione}, che la perturbazione prenda in input e restituisca come output delle stringhe. Sono quindi definiti due moduli, detti moduli speciali, che hanno lo scopo di convertire una stringa in una sequenza di token e viceversa:
\newcommand{\mto}{$M_{to}$}
\newcommand{\mde}{$M_{de}$}
\begin{itemize}
\item Modulo di tokenizzazione (\mto)
\item Modulo di detokenizzazione (\mde)
\end{itemize}

\subsubsection{Modulo di tokenizzazione}
Il modulo di tokenizzazione ha lo scopo di tokenizzare una qualsiasi stringa $s \in S$ in una lista $[t_1,...,t_n]$ di token:
\begin{equation}
tok: S \rightarrow [t_1,...,t_n]
\end{equation}
\paragraph{Esempio}
Data la frase $s_{ex} = \textit{"Nel mezzo del cammin di nostra vita"}$  
\begin{equation}
tok(s_{ex}) = \textit{['Nel', 'mezzo', 'del', 'cammin', 'di', 'nostra', 'vita']}
\end{equation}

\subsubsection{Modulo di detokenizzazione}
Il modulo di tokenizzazione ha lo scopo di detokenizzare, ovvero di eseguire il processo contrario alla tokenizzazione, una lista di token $[t_1,...,t_n]$ in una stringa $s \in S$.
\begin{equation}
detok: [t_1,...,t_n] \rightarrow S
\end{equation}
\paragraph{Esempio}
Data la sequenza di token $T_{ex} =$ \textit{['Nel', 'mezzo', 'del', 'cammin', 'di', 'nostra', 'vita']}.
\begin{equation}
detok(T_{ex}) = \textit{"Nel mezzo del cammin di nostra vita"}
\end{equation}



\subsection{Pipeline}
\label{sec:dst:pipeline}
Come accennato nelle sottosezioni precedenti, ogni modulo di perturbazione modella un diverso tipo di errore. In realtà, in un testo estratto tramite OCR sono presenti più tipologie di errori. \E\ quindi necessario utilizzare più moduli di perturbazione in sequenza per ottenere una perturbazione verosimile. Considerando che ogni modulo di perturbazione è considerabile come una funzione con dominio e codominio coincidenti $T$ (ogni elemento del dominio e codominio $T$ è una lista di token $[t_1,...,t_n]$ con $ n \in \mathbb{N} \wedge n \geqslant 0$), è possibile concatenare più moduli insieme tramite composizione:
\begin{equation}
m_k \circ m_{k-1} \circ m_{k-2} \circ ... \circ m_1
\end{equation}
dove ogni $m_i$ è istanza di un modulo di perturbazione. La funzione composta non può però trattare stringe, ma solo sequenze di token. Ciò è ovviabile attraverso l'aggiunta dei moduli speciali di tokenizzazione e detokenizzazione all'inizio e alla fine della composizione. La funzione risultante prende il nome di pipeline:
\begin{equation}
Pipeline: detok \circ m_k \circ m_{k-1} \circ m_{k-2} \circ ... \circ m_1 \circ tok
\end{equation}
Data quindi una stringa $s \in S$, la stringa perturbata $s\prime \in S$ è ottenuta come:
\begin{equation}
Pipeline(s) = s\prime
\end{equation}

\paragraph{Esempio}
Si definisca una pipeline $p_{ex}$ composta come indicato in \autoref{tab:dst_pex}.

\begin{table}[H]
\centering
\begin{tabular}{ccccc}
\textbf{Posizione} & \textbf{Nome istanza} & \textbf{Tipo Modulo} & \textbf{p} & \textbf{punct}\\ \hline
1	& $m_{to}$	& \mto	& n.d 	& n.d 	\\
2	& $m_{cr}$	& \mcr	& 0.3	& n.d 	\\
3	& $m_{ss}$	& \mss	& 0.1	& n.d 	\\
4	& $m_{pi}$	& \mpi	& 0.1	& , 	\\
5	& $m_{de}$	& \mde	& n.d 	& n.d 	\\
\end{tabular}
\caption{Struttura della pipeline $p_{ex}$}
\label{tab:dst_pex}
\end{table}

$p_{ex}$ corrisponde dunque alla seguente funzione, schematizzata anche in \autoref{fig:dst_pex}.
\begin{equation}
p_{ex} = m_{de} \circ m_{pi} \circ m_{ss} \circ m_{cr} \circ m_{to}
\end{equation}

\begin{figure}[H]
\centering
\includegraphics[width=\textwidth]{immagini/dataset/pex}
\caption{Schema della pipeline $p_{ex}$}
\label{fig:dst_pex}
\end{figure}

Data una stringa di esempio $s_{ex} = $ \textit{"Nel mezzo del cammin di nostra vita"}, è mostrata in seguito la sequenza dei passaggi che portano alla perturbazione della frase:
\begin{enumerate}
\setcounter{enumi}{-1}
\item ($s_{ex}$)\\
\textit{"Nel mezzo del cammin di nostra vita"}

\item $m_{to}(s_{ex})$\\
\textit{['Nel', 'mezzo', 'del', 'cammin', 'di', 'nostra', 'vita']}

\item $m_{cr}(m_{to}(s_{ex}))$\\
\textit{['Nel', \underline{'inezzo'}, 'del', \underline{'camniin'}, 'di', 'nostra', 'vita']}

\item $m_{ss}(m_{cr}(m_{to}(s_{ex})))$\\
\textit{['Nel', 'inezzo', 'del', \underline{'c a m n i i n'}, 'di', 'nostra', 'vita']}

\item $m_{pi}(m_{ss}(m_{cr}(m_{to}(s_{ex}))))$\\
\textit{['Nel', 'inezzo', 'del', 'c a m n i i n', 'di', \underline{','} , 'nostra', 'vita']}

\item $m_{de}(m_{pi}(m_{ss}(m_{cr}(m_{to}(s_{ex})))))$\\
\textit{"Nel inezzo del c a m n i i n di, nostra vita"}
\end{enumerate} 

Si sottolinea come questa sia solo una delle possibili perturbazioni di $s_{ex}$: essendoci nella perturbazione una componente aleatoria, ogni esecuzione di una pipeline su uno stesso input può dare risultati diversi.


\subsection{Superpipeline}
\label{sec:dst:superpipeline}
Una pipeline di perturbazione perturba il testo in modo uniforme: ogni frase perturbata con una pipeline $p$ ha una concentrazione di errori simile. Nell'osservazione di testi acquisiti tramite OCR si nota però come la distribuzione degli errori non sia uniforme: si possono incontrare sezioni prive di errori, ma anche parti di testo in cui è presente molto rumore. Ciò è spesso dovuto alle condizioni del documento originale, che in alcune parti può essere più degradato che in altre. Questo comporta che una sola pipeline non può simulare fedelmente gli errori OCR: è necessario variare l'intensità e la modalità di perturbazione per ottenere un risultato più verosimile.\\
Si introduce quindi il concetto di superpipeline. Date:
\begin{itemize}
\item Una lista $\textit{Ppl} = [p_1,...,p_n]$ dove ogni $p_i$ è una pipeline,
\item Una lista di pesi $W = [w_1,...,w_n]$ dove $w_i \in \mathbb{N}$ è il peso associato ad $p_i$,
\end{itemize}
è possibile definire una superpipeline come una funzione che, data una stringa $s \in S$, ne produce una versione perturbata tramite una pipeline $p_{rand}$ scelta casualmente da $\textit{Ppl}$. La probabilità che una pipeline $p_i$ venga scelta corrisponde a:
\begin{equation}
P(p_{rand} = p_i) = \frac{w_i}
{{\sum_{j=0}^{n}}w_j}
\end{equation}


\paragraph{Esempio}
Si definiscono le tre pipeline $p_{ex1},\ p_{ex1},\ p_{ex1}$, rispettivamente in \autoref{tab:dst_pex1}, \autoref{tab:dst_pex2}, \autoref{tab:dst_pex3}. 
\begin{table}[H]
\centering
\begin{tabular}{ccccc}
\textbf{Posizione} & \textbf{Nome istanza} & \textbf{Tipo Modulo} & \textbf{p} & \textbf{punct}\\ \hline
1	& $m_{to}$	& \mto	& n.d 	& n.d 	\\
2	& $m_{cr}$	& \mcr	& 0.5	& n.d 	\\
3	& $m_{de}$	& \mde	& n.d 	& n.d 	\\
\end{tabular}
\caption{Definizione di $p_{ex1}$}
\label{tab:dst_pex1}
\end{table}

\begin{table}[H]
\centering
\begin{tabular}{ccccc}
\textbf{Posizione} & \textbf{Nome istanza} & \textbf{Tipo Modulo} & \textbf{p} & \textbf{punct}\\ \hline
1	& $m_{to}$	& \mto	& n.d 	& n.d 	\\
2	& $m_{cr}$	& \mcr	& 0.3	& n.d 	\\
3	& $m_{hm}$	& \mhm	& 0.1	& n.d 	\\
4	& $m_{de}$	& \mde	& n.d 	& n.d 	\\
\end{tabular}
\caption{Definizione di $p_{ex2}$}
\label{tab:dst_pex2}
\end{table}

\begin{table}[H]
\centering
\begin{tabular}{ccccc}
\textbf{Posizione} & \textbf{Nome istanza} & \textbf{Tipo Modulo} & \textbf{p} & \textbf{punct}\\ \hline
1	& $m_{to}$	& \mto	& n.d 	& n.d 	\\
2	& $m_{cr}$	& \mcr	& 0.3	& n.d 	\\
3	& $m_{ss}$	& \mss	& 0.1	& n.d 	\\
4	& $m_{pi}$	& \mpi	& 0.1	& , 	\\
5	& $m_{de}$	& \mde	& n.d 	& n.d 	\\
\end{tabular}
\caption{Definizione di $p_{ex3}$}
\label{tab:dst_pex3}
\end{table}

Si definisce inoltre la lista dei pesi $W =[1,3,2]$. Quindi, calcolata la somma dei pesi $w_1  + w_2 + w_3 = 6$, si ha che:
\begin{itemize}
\item $p_{ex1}$ è usata con probabilità $w_1 / 6 = 1 / 6 = 0.17$
\item $p_{ex2}$ è usata con probabilità $w_2 / 6 = 3 / 6 = 0.50$
\item $p_{ex3}$ è usata con probabilità $w_3 / 6 = 2 / 6 = 0.33$
\end{itemize}

La superpipeline composta è quindi schematizzabile come mostrato in \autoref{fig:dst_supex}.

\begin{figure}[H]
\centering
\includegraphics[width=\textwidth]{immagini/dataset/supex}
\caption{Schema della superpipeline descritta nell'esempio}
\label{fig:dst_supex}
\end{figure}



\section{Configurazione}
\label{dst:configurazione}
Nelle precedenti sezioni sono stati descritti la metodologia di creazione del dataset e il funzionamento del sistema di perturbazione. In questa sezione sono invece descritti i parametri usati per il sistema di perturbazione e quindi per la creazione del dataset.

\subsection{Pipeline e Superpipeline}
\label{sec:dst_pipsup}
In questa sottosezione sono definite le superpipeline usate per la perturbazione del dataset e le pipeline che le compongono. Sono state stabilite 3 tipologie di superpipeline, ognuna delle quali ha 3 livelli di intensità:

\begin{itemize}
\item \textbf{Token superpipeline}: sono superpipeline che introducono solo dei word error (\autoref{sec:met_introduzione}), ovvero errori contenuti all'interno di un singolo token che non impattano la tokenizzazione. Le token superpipeline sono identificate dai codici T1, T2, T3, dove T1 e T3 sono rispettivamente la superpipeline che introduce meno errori e quella che ne introduce di più.

\item \textbf{Segmentation superpipeline}: sono superpipeline che introducono solo dei word segmentation error (\autoref{sec:met_introduzione}), ovvero che dividono o uniscono uno o più token e che impattano quindi la tokenizzazione. Le segmentation superpipeline sono identificate dai codici S1, S2, S3, dove S1 e S3 sono rispettivamente la superpipeline che introduce meno errori e quella che ne introduce di più.

\item \textbf{Mixed Pipeline}: sono superpipeline che introducono sia word error che word segmentation error. Le segmentation superpipeline sono identificate dai codici M1, M2, M3, dove M1 e M3 sono rispettivamente la superpipeline che introduce meno errori e quella che ne introduce di più.
\end{itemize}

In tutto quindi sono definite 9 superpipeline (T1, T2, T3, S1, S2, S3, M1, M2, M3): questo numero permette di avere una buona granularità tenendo comunque contenuti i tempi durante la fase di testing.

\subsubsection{Pipeline}
Le superpipeline appena descritte sono formate dalla combinazione di più pipeline. Sono definiti tre tipi di pipeline:
\begin{itemize}
\item Token pipeline
\item Segmentation pipeline
\item Mixed pipeline
\end{itemize}
ognuna di queste tipologie di pipeline introduce gli stessi tipi di errore delle omonime superpipeline. Per ogni tipologia di pipeline sono definite 3 pipeline, ognuna della quali ha li stessi moduli con probabilità diverse.

\paragraph{Token pipeline} Una token pipeline è formata dai moduli presenti nella seguente \autoref{tab:dst_tkpip}:

\begin{table}[H]
\centering
\begin{tabular}{cccc}
\textbf{Posizione} & \textbf{Tipo Modulo} & \textbf{punct}\\ \hline
1	& \mto	& n.d 	\\
2	& \mcr	& n.d 	\\
3	& \mde	& n.d 	\\
\end{tabular}
\caption{Moduli presenti in una token pipeline}
\label{tab:dst_tkpip}
\end{table}

In \autoref{tab:dst_tkpipdef} sono definite le tre token pipeline. Nella tabella, $p_i$ indica la probabilità associata al modulo in posizione $i$.

\begin{table}[H]
\centering
\begin{tabular}{cccc}
\textbf{Codice} & \boldmath{$p_1$} & \boldmath{$p_2$} & \boldmath{$p_3$}  \\ \hline
$tok_1$	& n.d	& 0.1	& n.d 	\\
$tok_2$	& n.d	& 0.3	& n.d 	\\
$tok_3$	& n.d	& 0.3	& n.d 	\\
\end{tabular}
\caption{Definizione delle tre token pipeline}
\label{tab:dst_tkpipdef}
\end{table}



\paragraph{Segmentation pipeline}
Una segmentation pipeline è formata dai moduli presenti nella seguente \autoref{tab:dst_sgpip}:

\begin{table}[H]
\centering
\begin{tabular}{cccc}
\textbf{Posizione} & \textbf{Tipo Modulo} & \textbf{punct}\\ \hline
1	& \mto	& n.d 	\\

2	& \mhm	& n.d 	\\
3	& \mps	& , (virgola)	\\
4	& \mss	& n.d 	\\
5	& \mpi	& . (punto)	\\
6	& \mpi	& , (virgola)	\\
7	& \mpi	& ' (apostrofo) 	\\

8	& \mde	& n.d 	\\
\end{tabular}
\caption{Moduli presenti in una segmentation pipeline}
\label{tab:dst_sgpip}
\end{table}

In \autoref{tab:dst_sgpipdef} sono definite le tre segmentation pipeline. Nella tabella, $p_i$ indica la probabilità associata al modulo in posizione $i$.

\begin{table}[H]
\centering
\begin{tabular}{ccccccccc}
\textbf{Codice} 
& \boldmath{$p_1$} 
& \boldmath{$p_2$} 
& \boldmath{$p_3$}
& \boldmath{$p_4$}
& \boldmath{$p_5$}
& \boldmath{$p_6$}
& \boldmath{$p_7$}
& \boldmath{$p_8$}
\\ \hline
$seg_1$	& n.d	& 0.001	& 0.001	& 0.0025	& 0.005	& 0.005	& 0.005	& n.d\\
$seg_2$	& n.d	& 0.001	& 0.002	& 0.008		& 0.025	& 0.025	& 0.025	& n.d\\
$seg_3$	& n.d	& 0.01	& 0.02	& 0.05		& 0.1	& 0.1	& 0.1	& n.d\\
\end{tabular}
\caption{Definizione delle tre segmentation pipeline}
\label{tab:dst_sgpipdef}
\end{table}


\paragraph{Mixed pipeline} Le mixed pipeline derivano dalla composizione di una token pipeline con una segmentation pipeline. Le mixed pipeline sono definite in \autoref{tab:dst_mixpipdef}.

\begin{table}[H]
\centering
\begin{tabular}{cc}
\textbf{Codice} & \textbf{Definizione} \\ \hline
$mix_1$ & $seg_1 \circ tok_1$ \\
$mix_2$ & $seg_2 \circ tok_2$ \\
$mix_3$ & $seg_3 \circ tok_3$ \\
\end{tabular}
\caption{Definizione delle tre mixed pipeline}
\label{tab:dst_mixpipdef}
\end{table}


\subsubsection{Superpipeline}
Date le pipeline appena definite, le superpipeline si compongono come mostrato nella \autoref{tab:dst_suppipdefall}: ad ogni superpipeline sono associati i pesi di ciascuna delle sue pipeline.

\begin{table}[H]
\centering
\begin{tabular}{cccccccccc}
\textbf{Codice}
& \boldmath{$tok_1$} 
& \boldmath{$tok_2$} 
& \boldmath{$tok_3$} 
& \boldmath{$seg_1$} 
& \boldmath{$seg_2$} 
& \boldmath{$seg_3$} 
& \boldmath{$mix_1$}
& \boldmath{$mix_2$} 
& \boldmath{$mix_3$} 
\\ \hline
T1	& 6	& 4	& 1	& /	& /	& /	& /	& /	& / \\
T2	& 2	& 8	& 1	& /	& /	& /	& /	& /	& / \\
T3	& 1	& 6	& 4	& /	& /	& /	& /	& /	& / \\
S1	& /	& /	& /	& 6	& 4	& 1	& /	& /	& / \\
S2	& /	& /	& /	& 2	& 8	& 1	& /	& /	& / \\
S3	& /	& /	& /	& 1	& 6	& 4	& /	& /	& / \\
M1	& /	& /	& /	& /	& /	& /	& 6	& 4	& 1 \\
M2	& /	& /	& /	& /	& /	& /	& 2	& 8	& 1 \\
M3	& /	& /	& /	& /	& /	& /	& 1	& 6	& 4 \\
\end{tabular}
\caption{Definizione delle nove superpipeline}
\label{tab:dst_suppipdefall}
\end{table}



\subsection{Dataset}
In questa sottosezione sono descritte le istanze del dataset create e i loro parametri. Sono ricordati i seguito i parametri configurabili per la creazione di un dataset:
\begin{itemize}
\item $l_{min}$: lunghezza minima di una frase nel dataset.
\item $l_{max}$: lunghezza massima di una frase nel dataset.
\item \textit{Lingue}: lista di lingue consentite nelle frasi nel dataset.
\item \textit{Superpipeline}: lista di superpipelines usate per la perturbazione.
\item \textit{Dimensione}: numero di elementi che formano il dataset, se esso viene ridotto
\end{itemize}

In \autoref{tab:dst_dstconfig} sono elencate le versione del dataset definiti e i loro parametri:

\newcommand{\dsta}{dst@50}
\newcommand{\dstb}{dst@100}

\begin{table}[H]
\centering
\begin{tabular}{cccccc}
\textbf{Codice} & \boldmath{$l_{min}$} & \boldmath{$l_{max}$} & \textbf{Lingue} & \textbf{Superpipeline} & \textbf{Dimensione}\\ \hline
\dsta & 8 & 50 & [it] & \tiny[T1, T2, T3, S1, S2, S3, M1, M2, M3]& 10000\\
\dstb & 20 & 100 & [it] & \tiny[T1, T2, T3, S1, S2, S3, M1, M2, M3]& 10000\\
\end{tabular}
\caption{Configurazione delle versioni del dataset}
\label{tab:dst_dstconfig}
\end{table}

Le due versioni del dataset differiscono solamente nella lunghezza massima e minima delle frasi. In questo modo sarà possibile testare quanto la lunghezza di una frase (e quindi il contesto intorno ad un errore) influisca sulle performance di correzione. Questi aspetti, insieme ai risultati dei testi, sono approfonditi nel \autoref{sec:test}.








\chapter{Metodologia di correzione}
\label{sec:metodologia}
\hl{\textbf{READ ME}}\\
\hl{Nello scrivere questo capitolo, faccio le seguenti assunzioni:\\
- Nel capitolo "dataset e perturbazione" sono gia' presenti le definizioni di termini come tokenizzare, token, detokenizzare ecc\\
- E' gia' presente una spiegazione comprensiva di cosa sia BERT nel capitolo sullo stato dell'arte}


In questo capitolo è descritta la metodologia del sistema di correzione.\\
Nella \autoref{sec:met_introduzione} vengono descritti gli obiettivi del processo di correzione e le criticità che lo contraddistinguono.
Nella \autoref{sec:met_panoramica} è presente una panoramica generale del processo di correzione, e sono introdotte le varie fasi che lo compongono. Nella \autoref{sec:met_errdet} e \autoref{sec:met_errcor} invece sono descritte più nel dettaglio le fasi di error detection e correction che compongono il processo.


\section{Introduzione}
\label{sec:met_introduzione}
Lo scopo del processo di OCR-Post Processing è quello di correggere e minimizzare gli errori introdotti dall'acquisizione di testo da immagini. Più in generale, data una frase contenente degli errori, lo scopo del processo di correzione è quello produrre in output la stessa frase senza errori. Nel fare ciò è inoltre necessario assicurarsi di non introdurne di nuovi.\\
La metodologia di correzione sviluppata, una volta identificati gli errori, fa uso del BERT Masked Language Modeling per produrre una serie di candidati per la correzione. Un approccio simile è utilizzato in \cite{OCRMaskFilling}, dove una combinazione di BERT e FastText riesce a produrre candidati corretti nel 70\% dei casi. L'approccio appena citato però non include le fasi di error detection e scelta del candidato corretto, che sono invece implementate nella metodologia proposta e applicate al dataset descritto nel \autoref{sec:dataset}.

\section{Processo}
\label{sec:met_processo}

\subsection{Panoramica Generale}
\label{sec:met_panoramica}
Il processo di correzione si articola in più fasi, alcune delle quali si ripetono più volte. \E\ possibile distinguere i seguenti passaggi:
\begin{itemize}
\item \textbf{Error detection}: in questa fase vengono individuati e contrassegnati gli errori all'interno della frase, se presenti.
\item \textbf{Error correction}: in questa fase si tenta la correzione degli errori individuati in precedenza.
\end{itemize}

Come mostrato in \autoref{fig:met_generale}, queste due fasi sono ripetute più volte. Ciò serve per sfruttare al massimo le caratteristiche del sistema di Error Correction, che si basa sul BERT Masked Language Modeling. Come spiegato più approfonditamente nella \autoref{sec:met_errcor}, questa funzione fa uso del contesto della frase e dell'intorno della parola da correggere per proporre dei candidati per la correzione da effettuare.
Il sistema di correzione, specialmente in caso di frasi contenenti molti errori, potrebbe non essere in grado di correggerli tutti. \E\ però possibile che, dopo aver corretto alcuni errori, il sistema sia in grado di correggerne altri grazie al maggior contesto che le correzioni hanno portato all'interno della frase.


\begin{figure}[H]
\centering
\includegraphics[width=\textwidth]{immagini/metodologia/generale}
\caption{Schema riassuntivo della Metodologia}
\label{fig:met_generale}
\end{figure}


\subsection{Error Detection}
\label{sec:met_errdet}

Lo scopo dell'error detection è quello di contrassegnare gli errori presenti all'interno della frase per la successiva fase di error correction. Dato che l'error correction corregge gli errori a livello di token, vengono contrassegnati per la fase successiva tutti i token contenenti errori. A tale scopo, è necessario in primis tokenizzare la frase di partenza. Successivamente tutti i token contenenti errori vengono marcati per la correzione.\\
Quanto appena spiegato è riportato nello schema in \autoref{fig:met_errdet}.

\begin{figure}[H]
\centering
\includegraphics[width=\textwidth]{immagini/metodologia/error_detection}
\caption{Schema del processo di error detection}
\label{fig:met_errdet}
\end{figure}

\subsection{Error Correction}
\label{sec:met_errcor}

Il processo di error correction si basa sul BERT Masked Language Modeling. Data una qualsiasi frase, è possibile sostituire una parola con la stringa \textit{[MASK]}, detta maschera. Dando in input la frase mascherata al modello BERT, esso produrrà una serie di parole (da qui in poi candidati) associate alla loro probabilità di corrispondere al token mascherato.

\paragraph{Esempio} Data la frase 
\begin{center}
\textit{"che assistono ragazze in difficoltà, le persone \underline{soie} e abbandonate, gli ammalati e gli anziani."}
\end{center}
la parola \textit{"soie"} sottolineata è stata individuata come errore. \E\ quindi necessario mascherala, per dare la frase in input al modello BERT. La frase diventa dunque:
\begin{center}
\textit{"che assistono ragazze in difficoltà, le persone [MASK] e abbandonate, gli ammalati e gli anziani."}
\end{center}
BERT produce quindi una lista di candidati, di cui sono riportati solo i primi 5:
\begin{itemize}
\item \textit{"sole"} con probabilità 0.42
\item \textit{"anziane"} con probabilità 0.28
\item \textit{"povere"} con probabilità 0.08
\item \textit{"care"} con probabilità 0.03
\item \textit{"disabili"} con probabilità 0.01
\end{itemize}
\ \\
Bisogna sottolineare come la parola originale sia trasparente al modello BERT. Ciò significa che i candidati prodotti dal modello sono del tutto indipendenti dalla parola originale, e sono inferite unicamente dal contesto derivato dal resto della frase.\\
La fase di error correction inizia con i token contrassegnati come errore nella fase precedente. Si procede mascherando il primo token errato all'interno della frase. Siccome BERT necessità di una frase, e non di un'insieme di token, è necessario detokinizzare la frase. Fatto ciò è possibile produrre i candidati per la correzione, come mostrato in \autoref{fig:met_errgen}.

\begin{figure}[H]
\centering
\includegraphics[width=\textwidth]{immagini/metodologia/generazione_candidati}
\caption{Schema del processo di generazione dei candidati}
\label{fig:met_errgen}
\end{figure}


Fra i vari candidati proposti da BERT ne viene scelto solamente uno, che viene determinato sulla base di alcuni criteri che tengono conto anche del token originale contente l'errore. Può però accadere che nemmeno il candidato sia la correzione adatta: si pensi al caso in cui l'error detection contrassegna un token corretto come errore, o ad una frase in cui bert non produce la parola corretta fra i candidati. Ogni candidato scelto deve quindi passare un ulteriore filtro, che ha lo scopo di distinguere queste evenienze. Nel caso il candidato non passi il filtro, il sistema ignora la correzione; in caso contrario, la correzione viene sostituita al token mascherato. Quanto appena descritto è rappresentato nello schema in \autoref{fig:met_errgen}.
\begin{figure}[H]
\centering
\includegraphics[width=\textwidth]{immagini/metodologia/scelta_candidati}
\caption{Schema del processo di scelta dei candidati}
\label{fig:met_errgen}
\end{figure}
Il processo appena descritto si ripete per ogni errore contrassegnato durante la fase di error detection. Una volta completata la correzione dell'ultimo token errato, la fase di error correction può dirsi conclusa.


%\begin{figure}[H]
%\centering
%\includegraphics[width=\textwidth]{immagini/metodologia}
%\caption{blabla}
%\label{fig:met_}
%\end{figure}




\chapter{Test e Risultati}
\label{sec:test}
In questa sezione sono illustrate la metodologia di test, le metriche utilizzate e risultati ottenuti dai test svolti.\\
Nella sezione \autoref{sec:test_intro} sono illustrati gli obiettivi dei test e la metodologia con i quali sono svolti. Nella \autoref{sec:test_metriche} sono descritte le metriche usate per svolgere i test. Nella \autoref{sec:test_risultati} sono riportati i risultati ottenuti dai test.

\section{Introduzione}
\label{sec:test_intro}
La metodologia di test adottata ha lo scopo di valutare automaticamente e in modo ripetibile le prestazioni del sistema di correzione. Tale metodologia è utile durante lo sviluppo del sistema di correzione, in quanto rende possibile quantificare l'impatto delle modifiche apportate al funzionamento o alla configurazione.\\
Inoltre, i test svolti in questo capitolo rendono possibile:
\begin{itemize}
\item Valutare le performance del sistema su vari tipi e intensità di errore. Come illustrato nel \autoref{sec:dataset}, per ogni frase perturbata nel dataset, sono presenti 9 versioni perturbate con diverse superpipeline. Valutando le performance di correzione sulle diverse tipologie di frasi, è possibile determinare come si comporta il sistema di correzioni di fronte a diversi tipi di errore.

\item Confrontare le performance del sistema di correzione su diverse versioni del dataset. Nel \autoref{sec:dataset} sono infatti state definite due versioni del dataset, \dsta\ e \dstb, che differiscono per la lunghezza delle frasi. Portare a termine i test su entrambe le versioni permette quindi di capire quanto la lunghezza delle frasi influisca sulle prestazioni del sistema di correzione.

\item Confrontare il sistema di correzione sviluppato con altri sistemi di OCR post-processing per avere un riscontro sul quale meglio valutare i risultati ottenuti. A tale scopo, si utilizza come confronto la metodologia di correzione proposta in \cite{hamalainen2019paft}, già illustrata nella \autoref{sec:arte_nmt}; l'approccio utilizzato è però leggermente modificato, e utilizza una combinazione di FastText e Word2Vec per trovare le parole semanticamente simili a quelle errate (a differenza del solo Word2Vec come nel paper originale). Per distinguere il metodo sviluppato da quello usato come confronto, si ci riferirà con:
	\begin{itemize}
	\item \textbf{Bert} al metodo sviluppato e oggetto di questa tesi.
	\item \textbf{Ftwv} al metodo appena descritto usato come confronto.
	\end{itemize}

\end{itemize}

\paragraph{Metodologia} Sono dati:
\begin{itemize}
\item L'insieme $D = \{\textit{\dsta}, \textit{\dstb} \}$ dei dataset
\item L'insieme $M = \{\textit{Bert}, \textit{Ftwv}\}$ dei metodi di correzione
\end{itemize}
Ogni frase $f$ in \dsta\ o \dstb contiene il testo originale (\textit{text}), e le sue versioni perturbate nel campo \textit{perturbed}. Dato il metodo $m \in M$ che si vuole valutare, e data una delle versioni perturbate $v$ della frase nel campo \textit{perturbed}, si produce la tripla in \autoref{tab:test_tripla}.

\begin{table}[H]
\centering
\begin{tabular}{cc}
\textbf{Nome Campo} & \textbf{Descrizione} \\
\hline
$t_{or}$ & Testo originale (\textit{text)}\\
$t_{pe}$ & Testo perturbato ($v$)\\
$t_{co}$ & $v$ corretta con il metodo $m$
\end{tabular}
\caption{Tripla contenente la frase corretta con uno dei metodi in $M$}
\label{tab:test_tripla}
\end{table}
\noindent
Dato un dataset $d \in D$, si producono 9 di queste triple per ogni frase $f \in d$, una per ognuna delle versioni perturbate presenti nel campo \textit{perturbed} di  $f$. Le triple sono raggruppate in 9 insiemi (T1, T2, T3, S1, S2, S3, M1, M2, M3) in base alla superpipeline di riferimento del campo $t_{pe}$. Su ognuno di questi insiemi è quindi possibile calcolare le metriche illustrate nella \autoref{sec:test_metriche}.

\section{Metriche}
\label{sec:test_metriche}
La valutazione della correzione avviene attraverso le metriche definite in questa sezione. Le metriche definite sono:
\begin{enumerate}
\item Errori corretti per errore presente ($C/P$).
\item Errori introdotti per frase ($In/Ch$).
\item Errori introdotti per ogni errore corretto ($I/C$).
\item Riduzione nella distanza di Levenshtein. ($LDR$).
\item Distanza di Levenshtein totale ($LDT$).
\end{enumerate}
\noindent
Prima di spiegare nel dettaglio le metriche introdotte, è spiegato il concetto di "errore delimitato" usato nelle prime tre metriche.

\paragraph{Errore delimitato} Siano date:
\begin{itemize}
\item $s$: una sequenza di testo considerata corretta.
\item $s\prime$: una sequenza di testo che differisce da $s$ per alcuni errori. 
\end{itemize}
Si definiscono "\textit{matching block}" tutte le sottosequenze coincidenti di $s$ e $s\prime$, attraverso le quali è anche possibile allineare le due sequenze. Si prendano ad esempio le sequenze \textit{"Adesso io preferisco"} e\textit{"Addcso io prioferisco"}; è possibile allinearne i matching block come segue:
\begin{center}
\noindent
\texttt{\underline{Ad} es \underline{so io pr} e\ \ \underline{ferisco}}\\
\texttt{\underline{Ad} dc \underline{so io pr} io \underline{ferisco}}
\end{center}
Si noti come i matching block siano stati sottolineati. Le sottosequenze che non sono comprese nei matching block, nella seconda frase dell'esempio \textit{"dc"} e \textit{"io"}, sono quelle che contengono le differenze fra le due frasi, che in questo caso corrispondono agli errori. Queste sottosequenze sono chiamate \textit{errori delimitati}.\\
Si prenda ad esempio la frase:
\begin{center}
\textit{"Adesso io preferisco parlare spontaneamente."}
\end{center}
In \autoref{tab:test_erroridel} sono date alcune versioni perturbate della stessa frase. Gli errori delimitati sono segnati fra parentesi quadre:

\begin{table}[H]
\centering
\begin{tabular}{cc}
\textbf{Esempio} & \textbf{Errori delimitati}\\
\hline
A[,]des[,]so io preferisco parlare spontaneamente. & 2 \\
Ad[dc]so io pr[io]ferisco parlare spontaneamente. & 2 \\
Ad[c]sso[.] io pr[c]f[c]risc parlare spontaneamente. & 3\\
\end{tabular}
\caption{Versioni perturbate della frase di esempio con gli errori delimitati segnati dalle parentesi quadre}
\label{tab:test_erroridel}
\end{table}
\noindent
Si noti come ogni errore delimitato contribuisca al conteggio del numeri di errori con un +1 indipendentemente dal numero di caratteri da cui è composto.

Data quindi una qualsiasi tripla $t$ appartenente a uno dei 9 insieme definiti in precedenza, e formata dai campi $t_{or}$, $t_{pe}$, $t_{co}$, sono date le seguenti definizioni:

\begin{itemize}
\item \textbf{Errori corretti:} numero di errori delimitati che sono in $t_{pe}$ ma non in $t_{co}$. Più informalmente, è il numero di errori che il sistema di correzione utilizzato è riuscito a correggere.
\item \textbf{Errori presenti:} numero di errori delimitati che sono in $t_{pe}$. Più informalmente, è il numero di errori che la superpipeline di perturbazione utilizzata ha generato in $t_{pe}$.
\item \textbf{Errori introdotti:} numero di errori delimitati in $t_{co}$ che non sono né in $t_{pe}$, né in $t_{or}$. Più informalmente, è il numero di errori che il sistema di correzione ha introdotto provando a correggere $t_{pe}$.
\end{itemize}

Data quindi una qualsiasi tripla $t$ appartenente a uno dei 9 insieme definiti in precedenza, sono definite le seguenti funzioni:
\begin{itemize}
\item $f_{co}(t)$: data una tripla $t$, ne restituisce il numero di errori corretti.
\item $f_{pr}(t)$: data una tripla $t$, ne restituisce il numero di errori presenti.
\item $f_{in}(t)$: data una tripla $t$, ne restituisce il numero di errori introdotti.
\end{itemize}
\noindent
Date queste definizioni, sono di seguito definite le metriche elencate in precedenza.


\paragraph{Errori corretti per errore presente ($C/P$)} Lo scopo di questa metrica è quello di calcolare la percentuale di errori che il sistema di correzione riesce a risolvere. Dato un insieme $I$ di triple $(\text{$t_{or}$, $t_{pe}$, $t_{co}$})$, è possibile definire $C/P$ come segue:
\begin{equation}
C/P = \frac{
    \sum_{t \in I} f_{co}(t)
}{
	\sum_{t \in I} f_{pr}(t)
}
\end{equation}
$C/P$ vale 0 se il sistema di correzione non corregge alcun errore, mentre vale 1 se il sistema di correzione li corregge tutti. \E\ quindi chiaro come più alto è il valore di $C/P$, migliori sono le performance di correzione.


\paragraph{Errori introdotti per frase ($In/Ch$)} Lo scopo di questa metrica è quello di quantificare gli errori che il sistema di correzione introduce. Dato un insieme $I$ di triple $(\text{$t_{or}$, $t_{pe}$, $t_{co}$})$, è possibile definire $In/Ch$ come segue:
\begin{equation}
In/Ch = \frac{
    \sum_{t \in I} f_{in}(t)
}{
	|I|
}
\end{equation}
La metrica è divisa per il numero caratteri nel dataset, in modo tale da permettere anche la comparazione fra dataset con lunghezze differenti. In questo caso, valori più bassi di $In/Ch$ sono desiderabili, in quanto si vuole evitare che il sistema di correzione non introduca nuovi errori.

\paragraph{Errori introdotti per ogni errore corretto ($I/C$)} Lo scopo di questa metrica è quello di misurare il rapporto fra gli errori che il sistema di correzione introduce durante la correzione, e gli errori presenti che invece vengono corretti. Dato un insieme $I$ di triple $(\text{$t_{or}$, $t_{pe}$, $t_{co}$})$, è possibile definire $I/C$ come segue:
\begin{equation}
I/C = \frac{
    \sum_{t \in I} f_{in}(t)
}{
	\sum_{t \in I} f_{co}(t)
}
\end{equation}
In questo caso, a valori più bassi di $I/C$ corrispondono performance migliori del sistema di correzione. Più precisamente:
\begin{itemize}
\item Se $I/C \geq 1$, significa che il sistema di correzione sta introducendo più errori di quanti ne corregge, peggiorando nel complesso il testo in input.

\item Se $0 < I/C < 1$, significa che il sistema di correzione sta correggendo più errori di quanti ne introduce, migliorando quindi il testo in input. Più il valore di $I/C$ approssima lo 0, migliori sono le performance di correzione.
\end{itemize}

\paragraph{Riduzione nella distanza di Levenshtein ($LDR$)}
Lo scopo di questa metrica è quello di misurare quanto il processo di correzione avvicina la frase perturbata a quella originale. Si misura quindi la differenza in distanza di Levenshtein fra frase originale e frase perturbata e fra frase originale e frase corretta. Lo scarto fra queste distanze indica la riduzione in distanza. Dato un insieme $I$ di triple $(\text{$t_{or}$, $t_{pe}$, $t_{co}$})$, è possibile definire $LDR$ come segue:
\begin{equation}
LDR = \frac{
\sum_{t \in I}d(t_{or},t_{pe}) - d(t_{or},t_{co})}{\sum_{t \in I}|t_{or}|}
\end{equation}
Il valore ottenuto divisa per la somma dei caratteri totali contenuti nell'insieme I. In questo modo la differenza è espressa nel rapporto riduzione su numero di caratteri, consentendo così di confrontare misurazioni su dataset differenti. Per questa metrica valori maggiori indicano migliori performance del sistema di correzione.


\paragraph{Distanza di Levenshtein totale ($LDT$)}
Lo scopo di questa metrica è quello di misurare la distanza totale fra le frasi originali e quelle corrette. Dato un insieme $I$ di triple $(\text{$t_{or}$, $t_{pe}$, $t_{co}$})$, è possibile definire $LDT$ come segue:
\begin{equation}
LDT = \frac{
\sum_{t \in I}d(t_{or},t_{co})}{\sum_{t \in I}|t_{or}|}
\end{equation}
Il valore ottenuto divisa per la somma dei caratteri totali contenuti nell'insieme I. In questo modo la differenza è espressa nel rapporto riduzione su numero di caratteri, consentendo così di confrontare misurazioni su dataset differenti. Per questa metrica valori più bassi indicano migliori performance del sistema di correzione.


\section{Risultati}
\label{sec:test_risultati}

Lo scopo dei test in questa sezione è quello di valutare le performance del sistema di correzione sviluppato e confrontarle a quelle del sistema Ftwv illustrato precedentemente. Si vuole inoltre misurare l'impatto della lunghezza della frasi da correggere sulle performance di correzione. Per fare ciò, si eseguiranno i test sulle seguenti configurazioni:
\begin{itemize}
\item \textbf{Bert@50}: sistema di correzione Bert con dataset \dsta.
\item \textbf{Bert@100}: sistema di correzione Bert con dataset \dstb.
\item \textbf{Ftwv@50}: sistema di correzione Ftwv con dataset \dsta.
\item \textbf{Ftwv@100}: sistema di correzione Ftwv con dataset \dstb.
\end{itemize}





\paragraph{Errori corretti per errore presente ($C/P$)}
In \autoref{fig:test_res_cp_gra} e \autoref{fig:test_res_cp} sono riportate le misurazioni per la metrica $C/P$. Dati i risultati, è possibile fare le seguenti osservazioni:
\begin{itemize}
\item Per ogni superpipeline, le configurazioni che usano frasi più lunghe ottengono risultati migliori rispetto alle controparti che utilizzano frasi da 50 caratteri. Questo fatto supporta l'ipotesi che la presenza di contesto più amplio attorno a un errore possa aiutare durante la correzione.

\item Il sistema di correzione sviluppato ottiene performance significativamente maggiori nelle token superpipeline. Questo era il risultato atteso, in quanto la correzione dei word error è stato il focus principale durante lo sviluppo del sistema. \hl{Possibili altre osservazioni?}

\end{itemize}

\begin{table}[H]
\centering
\begin{tabular}{c|cc|cc}
\textbf{Superpipeline} & \textbf{Bert@50} &  \textbf{Ftwv@50} & \textbf{Bert@100} & \textbf{Ftwv@100}\\
\hline
M1& 0.29& 0.17& 0.32& 0.22\\
M2& 0.28& 0.18& 0.33& 0.25\\
M3& 0.23& 0.16& 0.26& 0.24\\
S1& 0.26& 0.14& 0.27& 0.15\\
S2& 0.25& 0.17& 0.25& 0.17\\
S3& 0.27& 0.16& 0.28& 0.16\\
T1& 0.33& 0.16& 0.4& 0.24\\
T2& 0.34& 0.18& 0.4& 0.27\\
T3& 0.29& 0.17& 0.37& 0.28\\
\end{tabular}
\caption{Misurazioni di $C/P$ su tutti gli insiemi di frasi}
\label{fig:test_res_cp_gra}
\end{table}

\begin{figure}[H]
\centering
\includegraphics[width=\textwidth]{immagini/test/cp}
\caption{Misurazioni di $C/P$ su tutti gli insiemi di frasi}
\label{fig:test_res_cp}
\end{figure}

\paragraph{Errori introdotti per frase ($In/Ch$)}
In \autoref{fig:test_res_if_gra} e \autoref{fig:test_res_if} sono riportate le misurazioni per la metrica $In/Ch$. Dati i risultati si possono fare le seguenti osservazioni:
\begin{itemize}
\item In questo caso la lunghezza delle frasi non influenza i risultati ottenuti, che sono equiparabili a parità di sistema di perturbazione.

\item Si nota un amplio scarto fra i due sistemi di correzione: ciò in parte è attribuibile all'efficacia dello stadio di validazione durante la correzione nel sistema bert. Non è però possibile escludere che il sistema Ftwv sia particolarmente prono all'introduzione di errori durante la correzione.
\end{itemize}

\begin{table}[H]
\centering
\begin{tabular}{c|cc|cc}
\textbf{Superpipeline} & \textbf{Bert@50} &  \textbf{Ftwv@50} & \textbf{Bert@100} & \textbf{Ftwv@100}\\
\hline
M1& 0.0051& 0.0226& 0.0055& 0.0245\\
M2& 0.0056& 0.0217& 0.0055& 0.0235\\
M3& 0.0063& 0.0206& 0.0076& 0.024\\
S1& 0.0044& 0.024& 0.0042& 0.0236\\
S2& 0.0043& 0.0235& 0.0042& 0.023\\
S3& 0.0043& 0.0219& 0.0046& 0.0212\\
T1& 0.0049& 0.0238& 0.0048& 0.0255\\
T2& 0.0051& 0.0232& 0.005& 0.0253\\
T3& 0.0058& 0.023& 0.0056& 0.0259\\
\end{tabular}
\caption{Misurazioni di $In/Ch$ su tutti gli insiemi di frasi}
\label{fig:test_res_if_gra}
\end{table}

\begin{figure}[H]
\centering
\includegraphics[width=\textwidth]{immagini/test/if}
\caption{Misurazioni di $In/Ch$ su tutti gli insiemi di frasi}
\label{fig:test_res_if}
\end{figure}










\paragraph{Errori introdotti per ogni errore corretto ($I/C$)}
LOL

\begin{table}[H]
\centering
\begin{tabular}{c|cc|cc}
\textbf{Superpipeline} & \textbf{Bert@50} &  \textbf{Ftwv@50} & \textbf{Bert@100} & \textbf{Ftwv@100}\\
\hline
M1& 0.36& 2.79& 0.35& 2.3\\
M2& 0.32& 2.01& 0.26& 1.49\\
M3& 0.29& 1.37& 0.3& 1.05\\
S1& 1.02& 10.26& 0.91& 9.38\\
S2& 0.89& 6.75& 0.83& 6.4\\
S3& 0.45& 3.76& 0.44& 3.53\\
T1& 0.4& 3.98& 0.32& 2.8\\
T2& 0.32& 2.74& 0.26& 1.96\\
T3& 0.28& 1.97& 0.22& 1.36\\
\end{tabular}
\caption{Misurazioni di $I/C$ su tutti gli insiemi di frasi}
\label{fig:test_res_ic_gra}
\end{table}

\begin{figure}[H]
\centering
\includegraphics[width=\textwidth]{immagini/test/ic}
\caption{Misurazioni di $I/C$ su tutti gli insiemi di frasi}
\label{fig:test_res_ic}
\end{figure}









\paragraph{Riduzione nella distanza di Levenshtein ($LDR$)}
LOL


\begin{table}[H]
\centering
\begin{tabular}{c|cc|cc}
\textbf{Superpipeline} & \textbf{Bert@50} &  \textbf{Ftwv@50} & \textbf{Bert@100} & \textbf{Ftwv@100}\\
\hline
M1& 0.01& -0.04& 0.01& -0.04\\
M2& 0.01& -0.04& 0.02& -0.03\\
M3& 0.02& -0.03& 0.03& -0.02\\
S1& -0.0& -0.05& -0.0& -0.05\\
S2& -0.0& -0.04& -0.0& -0.04\\
S3& 0.0& -0.04& 0.0& -0.04\\
T1& 0.01& -0.05& 0.01& -0.05\\
T2& 0.01& -0.04& 0.02& -0.04\\
T3& 0.02& -0.04& 0.02& -0.04\\
\end{tabular}
\caption{Misurazioni di $LDR$ su tutti gli insiemi di frasi}
\label{fig:test_res_ldr_gra}
\end{table}

\begin{figure}[H]
\centering
\includegraphics[width=\textwidth]{immagini/test/ldr}
\caption{Misurazioni di $LDR$ su tutti gli insiemi di frasi}
\label{fig:test_res_ldr}
\end{figure}









\paragraph{Distanza di Levenshtein totale ($LDT$)}
LOL

\begin{table}[H]
\centering
\begin{tabular}{c|cc|cc}
\textbf{Superpipeline} & \textbf{Bert@50} &  \textbf{Ftwv@50} & \textbf{Bert@100} & \textbf{Ftwv@100}\\
\hline
M1& 0.06& 0.1& 0.05& 0.11\\
M2& 0.07& 0.12& 0.07& 0.12\\
M3& 0.12& 0.16& 0.11& 0.16\\
S1& 0.02& 0.06& 0.02& 0.06\\
S2& 0.02& 0.07& 0.02& 0.07\\
S3& 0.04& 0.08& 0.04& 0.08\\
T1& 0.04& 0.1& 0.04& 0.1\\
T2& 0.05& 0.11& 0.05& 0.11\\
T3& 0.08& 0.13& 0.07& 0.13\\
\end{tabular}
\caption{Misurazioni di $LDT$ su tutti gli insiemi di frasi}
\label{fig:test_res_ldt_gra}
\end{table}

\begin{figure}[H]

\section{Conclusioni}
\centering
\includegraphics[width=\textwidth]{immagini/test/ldt}
\caption{Misurazioni di $LDT$ su tutti gli insiemi di frasi}
\label{fig:test_res_ldt}
\end{figure}












% Il modello in lab si chiama Ftwv


\chapter{Analisi dell'errore}
\label{sec:analisi}
\hl{Fillare introduzione}

\section{Introduzione}
\label{sec:ana_intro}
Nel \autoref{sec:test} sono esposti i risultati dei test sul sistema di correzione. I test svolti valutano le performance complessive del sistema, ma non danno alcuna indicazione rispetto alle fasi che lo compongono. Identificare quali sono le fasi che più influenzano negativamente o positivamente le performance di correzione può essere utile per capire quali sono le componenti da migliorare con più priorità per lo sviluppo del sistema.\\
L'analisi presentata in questo capitolo riguarda nello specifico la correzione dei word error (\autoref{sec:met_introduzione}), e quindi del modulo di correzione token (\autoref{sec:met_tok_correct}). Come visto in precedenza, il modulo di correzione token opera in due fasi: error detection e error correction. L'analisi, che verte sulla fase di error correction, valuterà:
\begin{itemize}
\item La capacità del modello di BERT di produrre la soluzione per la correzione fra i vari candidati. La presenza della soluzione fra i candidati è condizione necessaria per poter applicare la giusta correzione: è quindi importante capire qual è il limite superiore che la generazione dei candidati pone al processo di correzione.

\item La capacità del sistema di correzione di scegliere il candidato corretto, se presente, fra quelli proposti dal modello BERT. Verrà valutata anche la capacità del sistema di evitare di apportare correzioni nel caso BERT non produca la soluzione fra i candidati.
\end{itemize}
La valutazione di questi due step del processo di correzione permetterà di stabilire quali sono i margini di miglioramento nella fase di error correction, considerando il limite superiore imposto dalla generazione dei candidati del modello BERT. Questo, confrontato con i risultati dei testi illustrati nel \autoref{sec:test}, permetterà di valutare anche il margine di errore della fase di error detection.


\section{Creazione di un dataset specifico}
\label{sec:ana_dst}
Come spiegato nell'introduzione, l'analisi presentata in questo capitolo verte unicamente sulla parte di error correction. \E\ quindi chiaro come la metodologia dell'analisi deve prevedere un modo per escludere l'error detection dal processo di correzione. Ciò implica che è necessario disporre frasi in cui gli errori sono in posizione nota, in modo da non dover fare error detection.

\paragraph{Metodologia} L'idea è quella di perturbare ulteriormente le frasi presenti in \dstb, introducendo un solo word error per frase. Il token perturbato non viene però reinserito nella frase, ma viene mascherato, permettendo così di conoscere la posizione dell'errore da correggere ed di avere una frase già pronta per la generazione dei candidati.\\
Ad esempio, data la frase già perturbata:
\begin{center}
\textit{"Sarà n e c e s s a r i o uno sforzo straordinario per mobilitare le risorse."}
\end{center}
si introduce un ulteriore word error, e si maschera la posizione dell'errore, ottenendo la seguente tripletta:
\begin{enumerate}
\item Frase: \textit{"Sarà n e c e s s a r i o uno sforzo [MASK] per mobilitare le risorse."}
\item Token originale: \textit{"straordinario"}
\item Token perturbato: \textit{"strnordinnrio"}
\end{enumerate}
Quanto appena spiegato informalmente è un processo divisibile nelle seguenti fasi:
\begin{enumerate}
\item Estrazione
\item Perturbazione
\end{enumerate}

\paragraph{Estrazione}
Lo scopo di questa fase è quello di ottenere una lista di frasi perturbate con diverse superpipeline. Nel contesto di questo capitolo si intende una frase come una stringa senza l'aggiunta di ulteriori metadati. Per ogni sample $s \in \dstb$ è possibile estrarre due tipi di frase:
\begin{itemize}
\item La frase originale non perturbata, presente nel campo \textit{text}.
\item Una frase perturbata scelta dal campo \textit{perturbed}. Ogni frase perturbata è identificata dalla superpipeline usata per la sua perturbazione.
\end{itemize}
\hl{Spiegare il perche del numero 35000?}
Si ottiene l'insieme delle frasi estratte $D_{extr}$ estraendo casualmente frasi da \dstb\ in modo da rispettare la distribuzione in \autoref{tab:ana_distr}:
\begin{table}[H]
\centering
\begin{tabular}{cccc}
\textbf{Tipo frase} & \textbf{Num. frasi} \\
\hline
text & 3500 \\
T1 & 3500 \\
T2 & 3500 \\
T3 & 3500 \\
S1 & 3500 \\
S2 & 3500 \\
S3 & 3500 \\
M1 & 3500 \\
M2 & 3500 \\
M3 & 3500 \\ 
\hline
\textbf{Totale} & \textbf{35000}
\end{tabular}
\caption{Distribuzione delle frasi estratte. Per "tipo" si intende \textit{text} se la frase non è perturbata; se invece la frase è perturbata il tipo corrisponde alla superpipeline utilizzata}
\label{tab:ana_distr}
\end{table}
Ogni frase è inoltre associata al suo tipo: in questo modo durante l'analisi sarà possibile valutare i risultati anche i base all'intensità della perturbazione.

\paragraph{Perturbazione}
Per ogni frase $f \in D_{extr}$ il primo passaggio nella fase di perturbazione consiste nel tokenizzare la frase. Si ottiene quindi una lista di token $T = [t_1,...,t_n]$. Questo passaggio si rende necessario in quanto si vuole introdurre un errore all'interno di un singolo token.\\
Fra lista di token ottenuta dalla tokenizzazione si individua un subset $P \subseteq T$ di token detti "perturbabili". Un token $t$ è considerato perturbabile se soddisfa le seguenti condizioni:
\begin{itemize}
\item Dato lo stesso vocabolario $V$ usato per l'error detection (\autoref{sec:met_tok_errdet}), $t \in V$.

\item $t$ è lungo almeno due caratteri.
\end{itemize}
Queste condizioni servono ad evitare di ri-perturbare un token che è già stato perturbato in precedenza o che deriva dallo split di un altro token. Si pensi ad esempio a una situazione in cui \textit{"papa"} viene perturbato come \textit{"p a p a"}: pur essendo \textit{"a"} presente nel vocabolario, non sarebbe corretto considerare \textit{"a"} come un token perturbabile.\\
Si sceglie quindi in modo casuale un solo token $t_p$ da perturbare dall'insieme $P$. Per perturbare $t_p$ si usa la funzione $f_{alternative}$ del modulo di characters replacement (\autoref{dst:modpert}). Quindi, si ha che:
\begin{equation}
t\prime_p = f_{alternative}(t_p)
\end{equation}
Successivamente si sostituisce $t_p$ nell'insieme $T$ con il token \textit{"[MASK]"}, ottenendo la lista di token $T\prime = [t_1,...,[MASK],...,t_n]$. Infine, si detokenizza $T\prime$ per ottenere la frase mascherata $f\prime$. \\
Alla fine di questa fase, per ogni frase $f \in D_{extr}$ è prodotta una tupla strutturata come in \autoref{tab:ana_tuplapert}:

\begin{table}[H]
\centering
\begin{tabular}{cc}
\textbf{Campo} & \textbf{Contenuto}\\ \hline
\textit{sent} & Frase originale mascherata ($f\prime$)\\
\textit{maskedTok} & Token che è stato mascherato ($t\prime_p$)\\
\textit{correctTok} & Token originale non perturbato ($t_p$)\\
\textit{pertSup} & Tipo di $f$ (\autoref{tab:ana_distr}) \\


\end{tabular}
\caption{Tupla prodotta per ogni frase $f$ dalla fase di perturbazione}
\label{tab:ana_tuplapert}
\end{table}
\noindent
L'insieme delle tuple risultanti ottenuto da questa fase è detto $D_{an}$.








\section{Analisi generazione dei candidati}
\label{sec:ana_bert}

\section{Analisi scelta dei candidati}
\label{sec:ana_cor}

\chapter*{Conclusioni}
\addcontentsline{toc}{chapter}{Conclusioni}  
\label{sec:conclusioni}
In questa tesi è stato sviluppato e proposto un approccio automatico al problema dell'OCR post processing, basato sull'uso di un modello BERT pre-allenato. Sono state definite una metodologia di test e delle metriche apposite per valutare le performance del sistema sviluppato. Data la difficoltà nel trovare un dataset con sequenze parallele di testo acquisito tramite OCR e testo corretto, è stato sviluppato un sistema per introdurre in maniera controllata rumore in sequenze di testo pulite. Ciò ha consentito di ottenere un dataset su cui è stato possibile eseguire i test stabiliti. I risultati ottenuti mostrano come il sistema sviluppato, a seconda del livello di rumore presente all'interno del testo, riesca a correggere fra il 25\% e il 40\% degli errori presenti, a fronte di una minima quantità di errori introdotti. \E\ stata infine eseguita un'analisi sulle prestazioni del sistema, che ha permesso di individuarne le maggiori criticità per eventuali sviluppi futuri.\\
Il lavoro svolto in questa tesi dimostra come l'utilizzo di un modello BERT pre-allenato sia un approccio percorribile per il problema dell'OCR post-processing, sebbene l'analisi dell'errore dimostri come il sistema sviluppato abbia ampi margini di miglioramento.

\paragraph{Sviluppi futuri} I risultati dei test e l'analisi dell'errore nei capitoli \ref{sec:test} e \ref{sec:analisi} fanno emergere alcune aree in cui è possibile migliorare il sistema sviluppato:

\begin{itemize}
\item I test hanno dimostrato come all'aumentare della lunghezza del testo in input il sistema produca risultati migliori. Potrebbe quindi essere opportuno usare frasi di 256 caratteri (limite superiore per BERT) e portare avanti la correzione con un approccio a sliding window sul testo. Questo approccio avrebbe il vantaggio di massimizzare la lunghezza delle frasi e di eseguire automaticamente più iterazioni su errori non corretti. Sarebbe però necessario un nuovo metodo per valutare le prestazioni del sistema su segmenti di testo più lunghi.

\item L'analisi dell'errore ha dimostrato come la parte di error correction del sistema agisca correttamente almeno nel 70\% (nel caso del tipo di frase con performance peggiori). Il sistema di correzione, nel migliore dei casi, corregge il 40\% degli errori presenti. Buona parte di questo scarto può essere attribuibile alla fase di error detection, che andrebbe riscritta rimpiazzando l'attuale sistema che fa semplicemente uso di un vocabolario.

\item I test fanno emergere come il sistema di correzione non abbia ottime performance sui word segmetation error. Sarebbe quindi opportuno esplorare altre soluzioni per questo tipo di errori. In questo ambito, l'analisi della letteratura suggerisce come approcci basati su NMT possano ottenere buoni risultati (\autoref{sec:arte_nmt}).
\end{itemize}

\bibliographystyle{unsrt}
\bibliography{../bibliografia}
\addcontentsline{toc}{chapter}{Bibliografia}


\end{document}


 
