
\title{Perturbazione di testo}
\author{
        Sebastiano Caccaro
}
\date{\today}


\documentclass[12pt]{article}
\usepackage[utf8]{inputenc}
\usepackage{amssymb}
\usepackage{amsmath}
\usepackage[italian]{babel}

\begin{document}
\maketitle


\section{Introduzione}
In questo documento è descritto il funzionamento del perturbatore di testo. Il perturbatore è un funzione che, dato un un determinato testo come input, ne produce una versione perturbata tramite l'introduzione alcuni errori. \\
Tali errori sono modellati attraverso delle funzioni, chiamate moduli, che vengono composte tra loro per creare una funzione detta pipeline.
Più formalmente:
\begin{equation}
\textit{Pipeline}: m_k \circ m_{k-1} \circ m_{k-2} \circ ... \circ m_0
\end{equation}
dove $\{m_0,...,m_k \}$ sono moduli.\\
Il perturbatore può essere visto come una combinazione di $n$ pipeline, dove ogni pipeline $P_i$ è associata ad un peso $w_i$. Il testo in input $x$ viene diviso in una sequenza di blocchi $\{ x_1,...,x_k \}$ dove per ogni blocco viene applicata una sola pipeline $P_i$ con una probabilità
\begin{equation}
p_{prt_i} =
\frac{w_i}
{{\sum_{j=0}^{n}}w_j}
\end{equation}
Inoltre, viene definito il parametro del perturbatore $s \in \{0...1\}$, detto stickyness. Tale parametro comporta che, dato un blocco $x_h$ a cui è applicata la pipeline $P_i$, al blocco $x_{h+1}$ sia applicata la stessa pipeline con probabilità:
\begin{equation}
p_{stick_i} =
s + (1-s)  p_{prt_i}
\end{equation}


\section{Moduli}
La perturbazione del testo avviene attraverso diverse unità chiamate moduli. Sono definiti tre tipi di moduli:
\begin{itemize}
	\item Moduli di tokenizzazione
	\item Moduli di detokenizzazione
	\item Moduli di perturbazione
\end{itemize}

\subsection{Moduli di tokenizzazione}
I moduli di tokenizzazione sono delle funzioni che hanno lo scopo di scomporre del testo in diversi token. Più formalmente un modulo di tokenizzazione è definito come segue:
\begin{equation}
T: x \mapsto \{ t_1,t_2,..., t_n  \}
\end{equation}
dove $t_i$ è un token.

\noindent
\textbf{Esempio}
\\
\textit{Input}: Nel mezzo del cammin di nostra vita mi ritrovai per una selva oscura,	ché la diritta via era smarrita.
\\
\textit{Output}: \{'Nel', 'mezzo', 'del', 'cammin', 'di', 'nostra', 'vita', 'mi', 'ritrovai', 'per', 'una', 'selva', 'oscura', ',', 'ché', 'la', 'diritta', 'via', 'era', 'smarrita', '.'\}


\subsection{Moduli di detokenizzazione}
I moduli di detokenizzazone sono delle funzioni che hanno lo scopo di mettere insieme una lista ordinata di token in un'unica stringa di testo. Sono definiti come segue:
\begin{equation}
D: \{ t_1,t_2,..., t_n  \} \mapsto x
\end{equation}

\noindent
\textbf{Esempio}
\\
\textit{Input}: \{'Nel', 'mezzo', 'del', 'cammin', 'di', 'nostra', 'vita', 'mi', 'ritrovai', 'per', 'una', 'selva', 'oscura', ',', 'ché', 'la', 'diritta', 'via', 'era', 'smarrita', '.'\}
\\
\textit{Output}: Nel mezzo del cammin di nostra vita mi ritrovai per una selva oscura,	ché la diritta via era smarrita.


\subsection{Moduli di perturbazione}
I moduli di perturbazione sono delle funzioni che hanno lo scopo di inserire gli errori all'interno del testo. Un modulo di perturbazione è definito come segue:
\begin{equation}
P: \{ t_1,t_2,..., t_n  \} \mapsto \{ t'_1,t'_2,..., t'_k  \}
\end{equation}

Ogni modulo $P$ è caratterizzato dai seguenti parametri:
\begin{itemize}
\item $f: \{ t_1,t_2,..., t_g  \} \mapsto \{ t'_1,t'_2,..., t'_h  \} $. Una funzione che prende in input gruppi di $g$ token e restituisce $h$ token perturbati.
\item $p$: probabilità che un dato gruppo di token $\{ t_1,t_2,..., t_g  \}$ sia perturbato dalla funzione $f$.
\end{itemize}

A seconda della funzione $f$, si possono distinguere varie categorie di moduli di perturbazione, che modellano diversi errori. Inoltre, alcuni moduli possono richiedere alcuni parametri aggiuntivi.

\subsubsection{Modulo di split}
Il modulo di split modella l'errore in cui le lettere di una parola vengono intermezzate da degli spazi. La funzione è definita come:
\begin{equation}
\textit{Split}: \{ t_1 \} \mapsto \{ t'_1 \}
\end{equation}
Essendo $t_1$ un token, e quindi una sequenza di caratteri, definiamo $t'_1$ come:
\begin{equation}
t'_1 = c + '\ ' \forall c \in t_1
\end{equation}
\textbf{Esempio}\\
\textit{Input}: \{\texttt{'cammin'}\}\\
\textit{Output}: \{\texttt{'c a m m i n'}\}

\subsubsection{Modulo di aggiunta di punteggiatura}
Il modulo di aggiunta di punteggiatura modella l'aggiunta di un segno di punteggiatura come il punto, la virgola o un apostrofo fra due token. La funzione è definita come:
\begin{equation}
\textit{AddPunct}: \{ t_1 \} \mapsto \{ t_1, \textit{punct} \}
\end{equation}
dove \textit{punct} è parametro della funzione \textit{AddPunct}.\\
\textbf{Esempio}\\
\textit{Input}: \{\texttt{'cammin'}\}\\
\textit{Output}: \{\texttt{'cammin',','}\}

\subsubsection{Modulo di unione con trattino}
Il modulo di unione con trattino modella l'unione di due token attraverso l'uso del carattere '-'. È definito come segue:
\begin{equation}
\textit{MergeHypen}: \{ t_1,t_2 \} \mapsto \{ t_1 + \textit{'-'} + t_2 \}
\end{equation}
\textbf{Esempio}\\
\textit{Input}: \{\texttt{'del','cammin'}\}\\
\textit{Output}: \{\texttt{'del-cammin'}\}


\subsubsection{Modulo di divisione con virgola}
Il modulo di divisione con virgola modella l'inserimento di una o più virgole all'interno di un token. È definito come segue:
\begin{equation}
\textit{SplitComma}: \{ t_1 \} \mapsto \{ f(t_1) \}
\end{equation}
Definendo un token $t$ come una sequenza di caratteri $\{c_1,...,c_n\}$, e definendo un numero casuale $x \in \mathbb{N} \wedge 1 \leqslant x < n$, è possibile formalizzare $f$ come:
\begin{equation}
f(t) = \begin{cases}
\{c1,...,c_x\} +\ ',' + f(\{c_{x+1},...,c_n \})&\text{se $x < n$}\\
\ '' 									&\text{se $x \geqslant n$}
\end{cases}
\end{equation}

\noindent
\textbf{Esempio}\\
\textit{Input}: \{\texttt{'cammin'}\}\\
\textit{Output}: \{\texttt{'ca,mm,in'}\}

\subsubsection{Modulo di rimpiazzo caratteri}
Il modulo di rimpiazzo caratteri modella lo scambio di una sequenza di caratteri con un'altra all'interno dello stesso token. È definito come segue:
\begin{equation}
\textit{SubChar}: \{ t_1 \} \mapsto \{ f(t_1) \}
\end{equation}
La funzione $f$ è caratterizzata dai seguenti parametri:
\begin{itemize}
\item Un insieme di sequenze di caratteri $\{s_1,...,s_n\}$ dove ad ogni sequenza $s_i$ è associata una probabilità $p_i$. Ogni sequenza $s_i$ presente in un token $t$ viene rimpiazzata da un'altra sottosequenza con probabilità $p_i$.
\item Ad ogni sequenza $s_i$ è associato un insieme di sequenze $\{r_{i1},...,r_{ik}\}$ dove ogni sequenza $s_{ij}$ è associata ad un peso $w_{ij}$.
\end{itemize}
Se ad ogni probabilità $p_i$ viene associata una variabile $X_i \sim Ber(p_i)$, e definisco $S$ come l'insieme di tutte le sequenze $s_i$ all'interno di $t$, è possibile definire $f$ come:

\begin{equation}
f(t,S) = \begin{cases}
t & \text{se $S = \emptyset$} \\
f(t,S \setminus \{s_i\}) & \text{se $S \neq \emptyset \wedge X_i = 0$} \\
f(\{c_1,...,c_g\},S) + sub(\{c_{g+1},...,c_h\}) + f(\{c_{h+1},...,c_n\},S)  
& \text{se $S \neq \emptyset \wedge X_i = 1$} \\
\end{cases}
\end{equation}

dove se $\{c_{g+1},...,c_h\}$ è la sequenza $s_i$, $sub(\{c_{g+1},...,c_h\})$ è una sola fra le sequenze $\{r_{i1},...,r_{ik}\}$  scelta con probabilità
\begin{equation}
p_{sub_{ij}} =
\frac{w_{ij}}
{{\sum_{h=0}^{k}}w_{ij}}
\end{equation}

\noindent
\textbf{Esempio}\\
\textit{Input}: \{\texttt{'cammin'}\}\\
\textit{Output}: \{\texttt{'oanimin'}\}





















\end{document}