
\title{Metodologia di test}
\author{
        Sebastiano Caccaro
}


\documentclass[12pt]{article}
\usepackage[italian]{babel}
\usepackage[utf8]{inputenc}
\usepackage{amssymb}
\usepackage{amsmath}
\usepackage{hyperref}
\usepackage{color,soul}

\newcommand{\E}{È}

\begin{document}
\maketitle
\newcommand{\cz}{$c$}
\newcommand{\cp}{$c\prime$}
\newcommand{\cs}{$c\prime\prime$}


\section{Introduzione}
Questo documento ha lo scopo di definire le metodologie di test per valutare l'efficacia di alcuni dei metodi di OCR post-processing presenti in letteratura. Il documento è così strutturato:
\begin{itemize}
\item Nella \autoref{sec:approcci} è descritto il dataset usato per il training e/o la valutazione dei vari approcci.
\item Nella \autoref{sec:approcci} viene fornita una breve overview per ognuno degli approcci scelti. Vengono inoltre definite le metologie di allenamento e/o di test per ognuno degli approcci.
\item Nella \autoref{sec:metriche} sono definite le misure e le metriche attraverso le quali gli approcci scelti verranno valutati.
\end{itemize}


\section{Dataset} \label{sec:dataset}
Il dataset usato è ricavato da un insieme di documenti proveniente dagli archivi vaticani. I documenti comprendono trascrizioni di discorsi, encicliche, lettere e altri tipi di testo che spaziano dal 6 Luglio 1439 al 22 Aprile 2021.\\
Ogni documento contiene dei metadati che specificano dettagli come data, lingua, titolo e autore. Inoltre, per ogni documento sono presenti sia il testo originale che lo stesso testo già diviso in paragrafi.\\
Il testo estratto può essere considerato corretto sia a livello di singolo token, che a livello di segmentazione. \\
Da questi documenti viene dunque costruito un dataset nelle seguenti fasi:
\begin{itemize}
\item Frammentazione
\item Perturbazione
\end{itemize}

\subsection{Frammentazione} \label{sec:spezzettamento}
La fase di frammentazione consiste nel ricavare da ogni paragrafo uno o più frammenti di testo che andranno a costituire dei data point nel dataset. Le regole per la frammentazione sono quelle descritte in  \cite{nematus}\bibliographystyle{unsrt}:
\begin{itemize}
\item Ogni frammento deve avere al più 50 caratteri;
\item Quando possibile, la frammentazione avviene su segni di punteggiatura che sanciscono la fine di una frase (.?!:;);
\item Se il frammento in questione risultasse più lungo di 50 caratteri, la frammentazione avviene sui numeri;
\item Se ancora il frammento risultassi più lungo di 50 caratteri, la frammentazione avviene al $50^o$ carattere.
\end{itemize}
Nel frammentare i paragrafi, quando possibile, viene favorita la creazione di frammenti con un numero di caratteri il più vicino possibile a 50.\\
Applicando questo approccio, è stato possibile ottenere 2827879 frammenti distinti.

\subsection{Perturbazione} \label{sec:perturbazione}
I frammenti fin'ora individuati sono considerati corretti, ovvero non contengono errori. Per testare la capacità dei vari approcci individuati di correggere errori OCR, è necessario associare ad ogni frammento corretto $c$, un frammento $c'$ così definito:
\begin{equation}
c' = f_{err}(c)
\end{equation}
dove $f_{err}$ è un funzione che introduce degli errori in $c$.
Per testare l'efficacia di ogni approccio su testi con diversi livelli di intensità di errore, vengono prodotte più versioni dello stesso dataset. Ogni versione del dataset è ottenuta utilizzando una diversa funzione $f_{err}$.\\
Utilizzando le definizioni di Pipeline e SuperPipeline date in (\hl{Inserire ref qui}) è possibile associare ad ogni versione del dataset una funzione $f_{err}$ corrispondente ad una diversa SuperPipeline.\\
Sono quindi individuate due categorie di Pipeline:
\begin{itemize}
\item \textbf{Di segmentazione}: sono pipeline che introducono unicamente errori che intaccano la segmentazione del testo. Sono composti dai seguenti moduli:
	\begin{itemize}
	\item Split
	\item AddPunct
	\item MergeHypen
	\item SplitComma
	\end{itemize}
	Esempio di perturbazione:\\
	\textit{Input}: Adesso io preferisco parlare spontaneamente.\\
	\textit{Output}: A,des,so io preferisco parlare spontaneamente.
\item \textbf{Di token}: sono pipeline che introducono unicamente errori che intaccano i singoli token, senza danneggiare la segmentazione. Sono composti dai seguenti moduli:
	\begin{itemize}
	\item SubChar
	\end{itemize}
	Esempio di perturbazione:\\
	\textit{Input}: Adesso io preferisco parlare spontaneamente.\\
	\textit{Output}: Adczso io prioferisco parlare spontaneamente.
\item \textbf{Miste}: sono pipeline che introducono sia errori a livello di token, che di segmentazione. Sono composte dalla concatenazione di una pipeline di segmentazione concatenata ad una pipeline di token.\\
	Esempio di perturbazione:\\
	\textit{Input}: Adesso io preferisco parlare spontaneamente.\\
	\textit{Output}: Adcsso. io prcfcrisc parlare spontaneamente.
\end{itemize}

Le Pipeline sono definite in \autoref{Tab:pipeline}, dove ogni ad ogni modulo è associata la sua probabilità.
\begin{table}[h!]
\centering
\begin{tabular}{cccccc}
\textbf{Nome} & \textbf{Split} & \textbf{AddPunct} & \textbf{MergeHypen} & \textbf{SplitComma} & \textbf{SubChar} \\
\hline
s1 	& 0.0025 	& 0.005 	& 0.001 	& 0.001 	& / 		\\	
s2 	& 0.008 		& 0.025 	& 0.001 	& 0.002 	& /		\\
s3 	& 0.05 		& 0.1 	& 0.01 	& 0.02 	& / 		\\ \hline
t1 	& /			& /		& /		& /		& 0.1	\\
t2 	& /			& /		& /		& /		& 0.3	\\
t3 	& /			& /		& /		& /		& 0.8	\\
\end{tabular}

\caption{Definizione di Pipeline}
\label{Tab:pipeline}
\end{table}
\\
Sono denotate con tx le pipeline di token, e con sx le pipeline di segmentazione. Maggiore è x, maggiore è l'intensità di errore delle pipeline.\\
\E\ quindi possibile ricavare le seguenti pipeline miste:
\begin{itemize}
\item $m1 = s1 + t1$
\item $m2 = s2 + t2$
\item $m3 = s3 + t3$
\end{itemize}


Sono poi definite le seguenti SuperPipeline. Ogni superpipeline è composta da una serie di pipeline, ad ognuna delle quali è associato un peso:
\begin{table}[h!]
\centering
\begin{tabular}{cccccccccc}
\textbf{Nome} & \textbf{s1} & \textbf{s2} & \textbf{s3} & \textbf{t1} & \textbf{t2} & \textbf{t3} & \textbf{m1} & \textbf{m2} & \textbf{m3}\\
\hline
S1	& 6	& 4	& 1	& /	& /	& / & / & / & / \\ 
S2	& 2	& 8	& 1	& /	& /	& / & / & / & / \\ 
S1	& 1	& 6	& 4	& /	& /	& / & / & / & / \\  \hline
T1	& /	& /	& /	& 6	& 4	& 1 & / & / & / \\ 
T2	& /	& /	& /	& 2	& 8	& 1 & / & / & / \\  
T3	& /	& /	& /	& 1	& 4	& 4 & / & / & / \\   \hline
M1	& /	& /	& /	& /	& /	& / & 6 & 4 & 1 \\ 
M2	& /	& /	& /	& /	& /	& / & 2 & 8 & 1 \\  
M3	& /	& /	& /	& /	& /	& / & 1 & 6 & 4 \\ 
\end{tabular}
\end{table}
\\
Sono quindi presenti 9 versioni diverse del dataset, ognuna delle quali è composta da coppie $(c,c\prime)$ dove  
$c' = f_{err}(c)$, con una diversa funzione $f_{err} \in \{ S1,S2,S3,T1,T2,T3,M1,M2,M3 \} $ a seconda della versione.
\section{Approcci} \label{sec:approcci}
\hl{Non mi e' chiaro se sia necessario inserire una overview generale per ogni approccio, comprendente di dettagli su come formattare i dati in input ecc. Al momento ho solo inserito una descrizione del processo in generale che mi serve per meglio definire le misure.}

Ognuno degli approcci precedenti può essere visto come una funzione $f_{corr}$, che, applicata ad un frammento perturbato \cp\, produce un frammento corretto \cs. Più formalmente:
\begin{equation}
	f_{corr}(c\prime) = c\prime\prime
\end{equation}
Ai fini della valutazione di ognuno dei metodi usati, ognuno di essi, dato un dataset $D$ formato da coppie (\cz, \cp), produce un dataset $D^+$ così definito:
\begin{equation}
	D^+ = \{		(c,c\prime,f_{corr}(c\prime))\ \forall\ (c,c\prime)	 \in D	\}
\end{equation}

\section{Misure e metriche di valutazione} \label{sec:metriche}

La notazione proposta nelle precedenti sezioni può essere riassunta come segue:
\begin{itemize}
\item \cz\ è un frammento considerato corretto appartenente al dataset $D$;
\item \cp\ è una versione perturbata di \cz, o meglio $f_{err}(c)$;
\item \cs\ è il candidato di correzione prodotto da uno degli approcci, ovvero $f_{corr}(f_{err}(c))$.
\end{itemize}
Sono definite informalmente le seguenti misure:
\begin{itemize}
\item \textit{Errori corretti (EC)}: numero di errori presenti in \cp\ che non sono presenti in \cs.
\item \textit{Errori introdotti (EI)}: numero di errori presenti in \cs\ che non sono presenti né in \cp, né in \cz. 
\end{itemize}Preso un frammento corretto \cp\ e un frammento \cp\ o \cs\ chiamato $d$, chiamo \textit{matching block} una la sottosequenza di caratteri presente sia in \cp\ che in $d$. Ogni istanza di \textit{matching block} in \cp\ è associata ad una sola istanza in $d$ e viceversa.\\
Un \textit{errore} è definito come una sottosequenza di $d$ che non appartiene a nessun \textit{matching block}.\\
Considerando la frase di partenza "Adesso io preferisco parlare spontaneamente.", sono riportati fra parentesi quadre nella tabella sottostante gli errori, ovvero i caratteri che non appartengono a nessun \textit{matching block}.

\begin{table}[h!]
\centering
\begin{tabular}{cc}

\textbf{Esempio} & \textbf{Numero di errori}\\
\hline
A[,]des[,]so io preferisco parlare spontaneamente. & 2 \\
Ad[dc]so io pr[io]ferisco parlare spontaneamente. & 2 \\
Ad[c]sso[.] io pr[c]f[c]risc parlare spontaneamente. & 3\\
\end{tabular}
\end{table}

\noindent
Definendo quindi $f_{ec}$ e $f_{ei}$ come:
\begin{equation}
	f_{ec}(c,c\prime,c\prime\prime) = \text{Numero di errori corretti dall'algortimo di correzione}
\end{equation}
\begin{equation}
	f_{ei}(c,c\prime,c\prime\prime) = \text{Numero di errori introdotti dall'algortimo di correzione}
\end{equation}
posso, per ogni algoritmo di correzione definire \textit{EC} e \textit{EI} come:
\begin{equation}
	EC = \sum_{(c,c\prime,c\prime\prime)\in D^+}f_{ec}(c,c\prime,c\prime\prime)
\end{equation}
\begin{equation}
	EI = \sum_{(c,c\prime,c\prime\prime)\in D^+}f_{ei}(c,c\prime,c\prime\prime)
\end{equation}
\E\ chiaro che l'efficacia di un approccio è maggiore quando massimizza \textit{EC}, minimizzando allo stesso tempo \textit{EI}.

\bibliographystyle{unsrt}
\bibliography{../bibliografia}





\end{document}