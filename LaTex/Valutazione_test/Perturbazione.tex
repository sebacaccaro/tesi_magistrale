
\title{Valutazione test}
\author{
        Sebastiano Caccaro
}


\documentclass[12pt]{article}
\usepackage[utf8]{inputenc}
\usepackage{amssymb}
\usepackage{amsmath}
\usepackage[italian]{babel}
\usepackage{color,soul}
\usepackage{float}

\begin{document}
\maketitle


\abstract{In questo documento esposti i risultati dei testi ottenuti con diversi livelli e tecniche di perturbazione. Sono inoltre usati diversi metodi di correzione o più versioni dello stesso metodo.}


\section{Metodi}
In questa sezione sono descritte le configurazioni di ogni test eseguito in termini di dataset e di metodo di correzione. Ogni codice è inoltre caratterizzato da un identificativo univoco. Ogni identificativo è composto come segue:
\begin{center}
	\texttt{xxxyyz}
\end{center}
dove:
\begin{itemize}
	\item xxx è una stringa di tre lettere che individua il metodo di correzione utilizzato.
	\item yy è un numero che indica una determinata configurazione dei parametri del metodo utilizzato.
	\item z è una lettera che indica un diverso metodo di perturbazione utilizzato nel dataset.
\end{itemize}

\section{Definizioni configurazioni}
\subsection{Metodi di correzione}
I metodi di correzione usati sono i seguenti:
\begin{itemize}
	\newcommand{\pgp}{pgp}
	\newcommand{\pgpfull}{Project-gender-politics}
	\item \textbf{\pgpfull\ (pgp)}: metodo sviluppato da Francesco Periti. \hl{Nome lasciato cosi per mancanza di alternative}. Sono presenti le seguenti configurazioni:
	      \begin{itemize}
		      \item \texttt{00}: ottenuta allenando i modelli word2vec e gensim con i parametri di default.
		      \item \texttt{01}: ottenuta allenando i modelli word2vec e gensim con i seguenti parametri:
		            \begin{itemize}
			            \item min\_count = 5
			            \item window = 5
			            \item vector\_size = 100
			            \item sample = 1e-4
			            \item negative = 20
			            \item epochs = 50
			            \item min\_alpha = 0.0001
			            \item alpha = 0.025
		            \end{itemize}
		      \item \texttt{02}: Come 01, ma i modelli word2vec e fasttext sono allenati anche sulle frasi non perturbate.
	      \end{itemize}
\end{itemize}



\subsection{Dataset}

I tipi di dataset utilizzati sono i seguenti:
\begin{itemize}
	\item \textbf{a}: gli errori di tokenizzazione sono introdotti attraverso il modulo \textit{CharSub}.
	\item \textbf{b}: gli errori di tokenizzazione sono introdotti attraverso il modulo \textit{TokenSub}.
\end{itemize}


\section{Metriche}


\section{Risultati}
I risultati per la metrica x... sono.

\begin{table}[H]
	\centering
	\begin{tabular}{cccccccccc}
		       & \textbf{T1} & \textbf{T2} & \textbf{T3} & \textbf{S1} & \textbf{S2} & \textbf{S3} & \textbf{M1} & \textbf{M2} & \textbf{M3} \\ \hline
		pgp00a & 0.139       & 0.150       & 0.160       & 0.108       & 0.122       & 0.132       & 0.145       & 0.154       & 0.153       \\
	\end{tabular}
\end{table}



































\end{document}